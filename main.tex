\documentclass[11pt]{article}
\pdfoutput=1
\usepackage[margin=1in,a4paper]{geometry}
\usepackage{amsthm,amssymb,amsmath}  
\usepackage{xspace,enumerate}
\usepackage[dvipsnames]{xcolor}
\usepackage[colorlinks=true,urlcolor=Blue,citecolor=Green,linkcolor=BrickRed]{hyperref}
\usepackage[capitalise]{cleveref}
\usepackage[utf8]{inputenc}
\usepackage{thmtools}
\usepackage{thm-restate}
\usepackage{authblk}
\usepackage{todonotes}


\def\dd{\mathinner{.\,.}}
\newcommand{\R}{\mathbb{R}}
\newcommand{\Z}{\mathbb{Z}}
\newcommand{\N}{\mathbb{N}}
\renewcommand{\O}{\mathcal{O}}
\newcommand{\tO}{\tilde{\mathcal{O}}}
\renewcommand{\phi}{\varphi}
\newcommand{\set}[1]{\left\lbrace #1 \right\rbrace}
\newcommand{\floor}[1]{\left\lfloor #1 \right\rfloor}
\newcommand{\bigset}[1]{\big \lbrace #1 \big \rbrace}
\newcommand{\Bigset}[1]{\Big \lbrace #1 \Big \rbrace}
\newcommand{\eq}[1]{\begin{align*} #1 \end{align*}}
\newcommand{\defproblem}[3]{\begin{center}\vspace{2mm}\noindent\fbox{\begin{minipage}{0.96\textwidth} #1 \\ {\bf{INPUT:}} #2 \\ {\bf{OUTPUT:}} #3 \end{minipage}} \vspace{2mm}\end{center}}


\usepackage{microtype} %if unwanted, comment out or use option "draft"
\usepackage{amsmath,amsfonts}
\usepackage[cmbtt]{bold-extra}
\usepackage[T1]{fontenc}
\usepackage[utf8]{inputenc}
\usepackage{thm-restate}
\usepackage{comment}
\usepackage{forest}
\usepackage{xspace}
\usepackage{enumerate}
\usepackage{makecell}
\usepackage{listings}
\usepackage{multirow}
\usepackage{diagbox}
\usepackage{todonotes}
\usepackage{thmtools}
\usepackage[ruled,noline,noend]{algorithm2e}
\usepackage[capitalise]{cleveref}
\usepackage{graphics,adjustbox}
\usepackage{tabularx}
\usepackage{amssymb}
\usepackage{epsfig}
\usepackage{verbatim}
\usepackage{braket}
\usepackage[noadjust]{cite}
\usepackage{xspace}
\usepackage{bold-extra}
\usepackage[margin=1in]{geometry}

\pagestyle{myheadings}


\title{Two-dimensional pattern matching with $k$ mismatches}
\author[1]{Jonas Ellert}
\author[2]{Paweł Gawrychowski}
\author[3]{Adam Górkiewicz}
\author[4]{Tatiana Starikovskaya}
\affil[1]{?}
\affil[2]{?}
\affil[3]{?}
\affil[4]{?}


\theoremstyle{plain}
\newtheorem{theorem}{Theorem}
\newtheorem{lemma}{Lemma}  
\newtheorem{fact}{Fact}
\newtheorem{corollary}[fact]{Corollary}  
\newtheorem{observation}{Observation}
\theoremstyle{definition}
\newtheorem{definition}{Definition}
\newtheorem{example}{Example}
\newtheorem{conjecture}{Conjecture} 
\newtheorem*{claim}{Claim}
\newtheorem{problem}{Problem}
\theoremstyle{remark}
\newtheorem*{remark}{Remark}


\sloppy

\DeclareMathOperator*{\X}{X}
\DeclareMathOperator*{\Y}{Y}
\DeclareMathOperator*{\Ext}{Ext}
\DeclareMathOperator*{\score}{score}
\DeclareMathOperator*{\Ham}{Ham}
\DeclareMathOperator*{\ID}{Id}
\DeclareMathOperator*{\BD}{BorderDis}
\DeclareMathOperator*{\dom}{dom}
\DeclareMathOperator*{\charrr}{C}

\newcommand{\jonas}[2][]{\todo[color=green!40, #1]{\textbf{J:} #2}}
\newcommand{\jonasi}[2][]{\jonas[inline, #1]{#2}}

% BEGIN DOCUMENT
\begin{document}

\date{}
\maketitle

\begin{abstract}
\end{abstract}

%\thispagestyle{empty}
%\clearpage
%\setcounter{page}{1}


\section{Related Work}

\paragraph{Hamming-Threshold PM in 1D}


\begin{itemize}
	\item Early solutions for pattern matching with mismatches use fast Fourier transform and run in $\O(n \sqrt{m \log m})$ time \cite{Abrahamson1987}. The first algorithms that explicitly exploit the threshold $k$ run in time roughly $\O(nk)$ \cite{Landau1986,Galil1986} and use ``kangaroo jumping'', a technique in which each text position is considered as a potential occurrence, and verified by jumping from mismatch to mismatch using a data structure for longest common extensions.
	\item Faster solution in $\O(n\sqrt{k \log k})$ or $\tO(n + k^3n/m)$ time \cite{Amir2004} by still using convolutions, but also exploiting periodicity and using counting arguments. The latter result was first improved to $\tO(n + k^2n/m)$ time \cite{Clifford2016a}, and then further refined to $\tO(n + kn/\sqrt{m})$ time \cite{Gawrychowski2018}, which is a smooth trade-off between the previous $\tO(n\sqrt{k})$ and $\tO(n + k^2n/m)$ bounds. If Monte-Carlo randomization is allowed, then the time can be slightly improved to $\O(n + kn\sqrt{(\log m) / m})$ \cite{Chan2020}. A significantly faster algorithm would imply unexpected consequences for the complexity of boolean matrix multiplication \cite{Gawrychowski2018}.
	\item The $k$-mismatch occurrences of the pattern either have a simple and exploitable structure, or the pattern is close to being periodic. This property is a powerful tool for pattern matching with mismatches, and was analyzed in \cite{Bringmann2019,Charalampopoulos2020a}.
\end{itemize}

\paragraph{Exact PM in 2D}

Assuming $n \times n$ text and $m \times m$ pattern, $m \leq n$.

\begin{itemize}
	\item Two-dimensional pattern matching dates back (at least) to the 1970s, when Richard Bird generalized the Knuth-Morris-Pratt algorithm \cite{Knuth1977} for one-dimensional pattern matching, achieving $\O(n^2 + m^2)$ time \cite{Bird1977}.
	\item By using multiple pattern matching on only $\O(n/m)$ rows of the text, Baeza-Yates and Régnier achieve average time complexity $\O(n^2 / m)$, using $\O(m^2)$ space and still running in $\O(n^2)$ time in the worst case \cite{Baeza-Yates1993}. This was followed by more results optimizing the average time complexity \cite{Tarhio1996,Kaerkkaeinen1999}.
	\item Crochremore at al.\ \cite{Crochemore1995} solve two-dimensional pattern matching space efficiently, using an $\O(m^2)$ time and $\O(\log m)$ space preprocessing followed by an $\O(n^2)$ time and constant space search phase.
	\item On a PRAM, the problem can be solved in constant time and $O(n^2)$ work \cite{Crochemore1998}, after preprocessing the pattern in $\O(\log \log m)$ time and $\O(m^2)$ work \cite{Cole1993}.
	\item Other works studying the problem are, e.g., \cite{Baker1978,Karp1987,Zhu1988,Zhu1989,Amir1994,Galil1996}
\end{itemize}


\paragraph{Hamming-Threshold PM in 2D} Assuming $n \times n$ text and $m \times m$ pattern, $m \leq n$, and threshold $k < m^2$.

\begin{itemize}
	\item First result in 1987 by Krithivasan and Sitalakshmi \cite{Krithivasan1987}, who show that $\tO(kmn^2)$ time can be achieved.
	\item Faster algorithm by Rank and Heywood runs in $\tO((k+m)n^2)$ time \cite{Ranka1991}.
	\item Current state of the art\jonas{check} is $\O(kn^2)$ \cite{Amir1991}
	\item More work \cite{Baeza-Yates1998,Park1998,Kaerkkaeinen1999} is aimed at optimizing the average time complexity. This has also been done for the setting in which arbitrary rotations of the pattern are allowed \cite{Fredriksson2002}.
	\item Dynamic and online algorithms \cite{Clifford2016}
	\item Measures like edit distance are more complicated to define for two-dimensional strings \cite{Baeza-Yates1998a}.
\end{itemize}

\paragraph{2D Periodicity}

\begin{itemize}
	\item Two-dimensional periodicity was introduced in \cite{Amir1992,Amir1998} and refined in \cite{Galil1996}. All periods (and witnesses for all non-periods) can be computed in $\O(n^2)$ time \cite{Cole2004}. Witnesses can be used to efficiently eliminate potential occurrences of the pattern. Since witness computation is costly for run-length encoded strings, Amir at al.\ \cite{Amir1997} show additional properties of two-dimensial periodicities that allow the elimination of occurrences without witnesses.
	\item Famous periodicity theorems have been generalized to two dimensions, for example the Fine-Wilf theorem \cite{Fine1965,Mignosi2003} and the Lyndon-Schützenberger theorem \cite{Lyndon1962,Gamard2017}.
	\item Smallest rectangular cover in $\O(n^2)$ time, all aperiodic covers in $\O(n^2 \log n)$ time \cite{Charalampopoulos2021}. Covers of two-dimensional strings were also considered in \cite{Gamard2019,Crochemore1998a}. 
	\item All tile covers in $\O(n^{2+\epsilon})$ time \cite{Radoszewski2022}.
	\item There can be $\Omega(n^3 \log n)$ tandem substrings (repeated rectangle, like square in a normal string) \cite{Apostolico2000}.
	\item All tandems can be computed in $\O(n^3 \log n)$ time \cite{Apostolico2005}.
	\item There can be $\Omega(n^2 \log n)$ two-dimensional runs (maximal repetitions) \cite{Gawrychowski2021}.
	\item There are at most $\O(n^2 \log^2 n)$ two-dimensional runs \cite{Charalampopoulos2020}.
	\item All runs can be computed in $\O(n^2 \log^2 n)$ time \cite{Amir2020}.

\end{itemize}


\paragraph{Other 2D Shannanigans}

\begin{itemize}
	\item A survey by Rytter \cite{Rytter2000} discusses compressed representations of two-dimensional strings and the computation on such representations without explicit decompression. If the text is compressed as a two-dimensional SLP, then merely deciding if a pattern is present is NP-complete with respect to the compressed size \cite{Berman2002}. If both text and pattern are compressed using a two-dimensional version of Lempel-Ziv compression, then exact matching can be done in $\O(n^2 + m^3)$ time, using additional working space proportional to the compressed representation of the pattern \cite{Amir2003}. If text and pattern are given in two-dimensional run-length encoding, then the time for exact matching is linear in the size of the compressed text, and the space is linear in the size of the compressed pattern \cite{Amir2003a}. If multiple patterns are given, then they can be compressed into a space efficient self-index that allows fast dictionary matching \cite{Neuburger2013}.
	\item In parametrized matching, pattern occurrences have to be isomorphic via a bijection on the effective alphabets \cite{Amir2006,Cole2014}. A key step for efficient parametrized matching in two dimensions is counting the number of distinct symbols in every $m \times m$ substring, which can be done in $\O(n^2)$ time \cite{Cole2014}.
	\item More work on structural elements in two-dimensionsal strings, e.g., frames (rectangles with matching outer columns and rows) \cite{Boneh2023}, Lyndon words \cite{Marcus2017}, motif patterns (allowing ``don't cares'', and satisfying special properties) \cite{Apostolico2008}, patterns with at least two occurrences \cite{Karp1972}, and palindromic substrings \cite{Mahalingam2019}.
	\item There is an efficient algorithm for matching a half-rectangular pattern (fixed height, but irregular shape) in a text, allowing a fixed number of mismatches, insertions and deletions \cite{Amir1995}.
\end{itemize}


% INTRODUCTION
\section{Introduction}

\newcommand{\hd}{\textsc{HD1D}\xspace}
\newcommand{\HD}{\textsc{HD2D}\xspace}

We consider the one-dimensional all-substring Hamming distance problem (\hd), where for a given text string $T$ of length $n$ and a string $P$ of length $m$ ($m < n$), we want to calculate the Hamming distance between $P$ and every fragment $T$ of length $m$.

We consider the two-dimensional all-substring Hamming distance problem (\HD), where for a given 2D string $T$ of size $n \times n$ and a string $P$ of size $m \times m$ ($m < n$), we want to calculate the Hamming distance between $P$ and every $m \times m$ fragment of $T$.

We also consider the bounded variants of \hd and \HD, where we are only required to calculate the distances which are not greater than $k$, for some parameter $k \in \Z^+$.


\begin{theorem}[Main result]\label{main result}
	Bounded \HD can be solved in $\tO((m^2 + mk^{5/4})n^2 / m^2)$ time.
\end{theorem}


% PRELIMINARIES
\section{Preliminaries}

For our purposes we will not use the standard definition of a two-dimensional string, where we associate it with a two-dimensional array of characters, and instead we will define it more broadly.
Although we will occasionally use the array notation, we will do it exclusively for $n \times m$ strings.
For any $n \in \Z^+$ we will denote $[n] = \set{0, \dots, n - 1}$.
We will use the terms point and vector interchangeably.
Our results hold under word-RAM model of computation.

\newcommand{\getchar}[1]{\charrr(#1)}
\newcommand{\pto}{\mathrel{\ooalign{\hfil$\mapstochar\mkern5mu$\hfil\cr$\to$\cr}}}
\renewcommand{\d}[1]{\dom(#1)}
\newcommand{\f}[1]{#1^\mathbf{f}}
\begin{definition}[Two-dimensional string]
	We define a \textbf{string} $S$ as a partial function $\Z^2 \pto \Sigma$ which maps some arbitrary set of integer points, denoted as $\d{S}$, to characters.
	For simplicity we will write $u \in S$ to denote that $u \in \d{S}$.
	We say that a string $S$ is \textbf{partitioned} into strings $R_1, \dots, R_\ell$ when the sets $\d{R_1}, \dots, \d{R_\ell}$ partition $\d{S}$ and $R_i(u) = S(u)$ for all $u \in R_i$.
	We call a string $S$ \textbf{monochromatic} when $S(u) = \sigma$ for every $u \in S$ for some $\sigma \in \Sigma$ and we will write $\getchar{S}$ to denote the value $\sigma$.
	We say that $S$ is $n \times m$ for some $n, m \in \Z^+$ when $\d{S} = [n] \times [m]$.
	Physically we represent a string as a list of point-character pairs.
\end{definition}


\begin{definition}[Shifting]
	For a set of points $V \subseteq \Z^2$ and a vector $u \in \Z^2$, we denote $V + u$ as $\set{v + u : v \in V}$.
	For a string $S$ and a vector $u \in \Z^2$ we denote $S + u$ as a string $R$ such that
	$\d{R} = \d{S} + u$ and $R(v) = S(v - u)$ for $v \in \d{R}$.
	Intuitively, we shift the set of points while maintaining their character values.
\end{definition}


\begin{definition}[Hamming distance]
	For a pair of strings $S, R$ we define
	$$ \Ham(S, R) = |\set{u : u \in \d{S} \cap \d{R}, S(u) \neq R(u)}|,$$
	which corresponds to the number of mismatches between $S$ and $R$.
\end{definition}


Under such notation, the \HD problem is equivalent to calculating the (bounded or unbounded) values of $ \Ham(P + q, T) $
for all $q \in \Z^2$ such that $\d{P + q} \subseteq \d{T}$ (so for $q \in [n - m + 1]^2$).


\begin{definition}[Don't care symbol]
	We define the \textbf{don't care} symbol as a special character which matches with every character.
	We will denote it with \texttt{?}.
	Unless stated otherwise, we assume it is not allowed in $\Sigma$ and in both \hd and \HD every character present in $T$ and $P$ matches only with itself.
\end{definition}


\newcommand{\x}[1]{#1.x}
\newcommand{\y}[1]{#1.y}
\newcommand{\h}[1]{\phi \times #1}
\newcommand{\s}[1]{\psi \times #1}
\begin{definition}[Vector operators]
	For any $u \in \Z^2$ we refer to its coordinates as $\x{u}, \y{u}$.
	For $u, v \in \Z^2$ we denote $u \cdot v = \x{u} \cdot \x{v} + \y{u} \cdot \y{v}$
	and $u \times v = \x{u} \cdot \y{v} - \y{u} \cdot \x{v}$.
	Note that alternatively $u \cdot v = |u||v| \cos \alpha$ and $u \times v = |u||v| \sin \alpha$ where $\alpha$ is the angle between $u$ and $v$.
\end{definition}


\newcommand{\Q}{\mathcal{Q}}
\begin{definition}[Quadrants]
	We define the four \textbf{quadrants} as
	\eq{
		\Q_1 &= (0, +\infty) \times [0, +\infty), \\
		\Q_2 &= (-\infty, 0] \times (0, +\infty), \\
		\Q_3 &= (-\infty, 0) \times (-\infty, 0], \\
		\Q_4 &= [0, +\infty) \times (-\infty, 0).
	}
\end{definition}


\section{One-dimensional generalizations}
In this section we explore some of the methods used for one-dimensional strings.
Specifically, as our goal is to generalize the solution for pattern matching with $k$ mismatches described in \cite{Gawrychowski2017}, we are especially interested in two-dimensional variants of the techniques that were used to solve the one-dimensional case.


\begin{theorem}[Instancing]\label{instancing}
	Consider an algorithm $\mathcal{A}$ which solves \HD (bounded or unbounded), but only when $2|n$ and $n \le \frac{3}{2}m$.
	If its running time is $\mathcal{T}(m)$, then the general case can be solved in $\O(\mathcal{T}(m) n^2 / m^2)$.
	\begin{proof}
		Let $r = \floor{m / 2}$ and let $n' = r + m - 1$ or $r + m$ if $r + m - 1$ is odd.
		We see that the set $N = [n']^2$ satisfies the conditions for the text domain.
		For any vector $q \in [n - m]^2$ we can find a vector $u$ such that $r|u.x, r|u.y$ and $q - u \in [r]^2$,
		so we have $\Ham(P + q, T) = \Ham(P + q - u, T_u)$ where $T_u$ is the restriction of $T - u$ to $N$.
		If $T - u$ is not defined for some $v \in N$, we can pad $T_u(v)$ with any character.
		We see that $\d{P + q - u} \subseteq N = \d{T_u}$.
		There are $\O(n^2 / m^2)$ possible vectors $u$ and we run $\mathcal{A}$ for every pair of $T_u$ and $P$.
	\end{proof}
\end{theorem}


\begin{theorem}[Kangaroo jumps]\label{kangaroos}
	Consider an $n \times n$ string $T$, $m \times m$ string $P$ and set of vectors $Q$ such that $\d{P + q} \subseteq \d{T}$ for every $q \in Q$.
	There exists an algorithm which calculates $ d_q = \Ham(P + q, T) $ for every $q \in Q$ in total time $\tO(n^2 + \sum_{q \in Q} d_q)$.
	\begin{proof}
		For the sake of clarity, we will temporarily switch to the classical array notation for strings.
		Let $T_0, \dots, T_{n - m}$ denote an array of two-dimensional strings (arrays) such that $T_k[0 \dd n - 1, 0 \dd m - 1] = T[0 \dd n - 1, k \dd k + m - 1]$.
		For every row $P[0], \dots, P[m - 1]$ of $P$ and every row $T_k[0], \dots, T_k[n - 1]$ of every $T_k$ we assign an integer identifier so that $\ID(P[i]) = \ID(T_k[j]) \Leftrightarrow P[i] = T_k[j]$ by using the KMR algorithm ([reference]) in $\tO(n^2)$.
		
		We use the approach described in [kangaroo reference].
		There exists a data structure (suffix array) which for a given one-dimensional array $S$ allows us to detect all mismatches between any given two of its subarrays of equal length.
		It can be built in $\tO(|S|)$ and the query time is $\tO(d + 1)$ where $d$ is the number of mismatches.
		We construct the suffix array for the concatenation of the following arrays:
		\begin{itemize}
			\item the rows $P[i]$ for every $i$,
			\item the rows $T[i]$ for every $i$,
			\item the array $\ID(P[0]) \ID(P[1]) \dots \ID(P[m - 1])$,
			\item the arrays $\ID(T_k[0]) \ID(T_k[1]) \dots \ID(T_k[n - 1])$ for every $k$,
		\end{itemize}
		the total length of which is $\O(n^2)$.
		Let us consider a problem of detecting mismatches between $P$ and some $T' = T[j \dd j + m - 1, k \dd k + m - 1]$.
		We can first find all row indices $i$ for which $P[i] \neq T'[i]$ by finding all mismatches between $\ID(P[0]) \dots \ID(P[m - 1])$ and $\ID(T_k[j]) \dots \ID(T_k[j + m - 1])$, which we do with query to the data structure.
		For every such $i$ we can then find all mismatches between $P[i]$ and $T'[i]$ by querying $P[i]$ and $T[i + j][k \dd k + m - 1]$.
		If the distance between $P$ and $T'$ is $d$, the first query takes $\tO(d + 1)$ operations and all subsequent queries take $\tO(d + 1)$ operations in total.
	\end{proof}
\end{theorem}


\begin{lemma}\label{sigman1d}
	$\hd$ with don't care symbols can be solved in $\tO(n|\Sigma|)$ by running $|\Sigma|$ instances of FFT.
\end{lemma}


\begin{lemma}\label{approx1d}
	There exists a $(1 + \varepsilon)$-approximate algorithm (introduced in \cite{Karloff1993}) which solves \hd with don't care symbols in $\tO(n)$.
\end{lemma}


\begin{theorem}\label{sigman2d}
	$\HD$ with don't care symbols can be solved in $\tO(n^2|\Sigma|)$.
	\begin{proof}
		We will again use the array notation.
		We construct one-dimensional strings $\bar{T}$ and $\bar{P}$ by concatenating subsequent rows $T[0], \dots, T[n - 1]$ of $T$ and rows $P[0], \dots, P[m - 1]$ of $P$ padded with don't care symbols:
		$$ \bar{T} = T[0] \ T[1] \ \dots \ T[n - 1], $$
		$$ \bar{P} = P[0] \ \texttt{?}^{n - m} \ P[1] \ \texttt{?}^{n - m} \ \dots \ \texttt{?}^{n - m} \ P[m - 1].$$
		We run the algorithm from \Cref{sigman1d}.
		The distance between $T[i \dd i + m - 1, j \dd j + m - 1]$ and $P$ is equal to the distance between $\bar{T}[in + j \dd in + j + nm - n + m - 1]$ and $\bar{P}$.
	\end{proof}
\end{theorem}


\begin{theorem}\label{approx2d}
	There exists a $(1 + \varepsilon)$-approximate algorithm which solves \HD with don't care symbols in $\tO(n^2)$.
	\begin{proof}
		Identical to \Cref{sigman2d}, but we use the algorithm from \Cref{approx1d} instead of \Cref{sigman1d}.
	\end{proof}
\end{theorem}


The same reduction as in \Cref{sigman2d} can be applied for every \hd solution which allows don't care symbols.
Unfortunately, the most effective known algorithms for bounded \hd rely on periodicity (\cite{Clifford2015}, \cite{Gawrychowski2017}) and inherently do not allow don't care symbols, thus, they cannot be easily generalized.


\begin{observation}[Don't care padding]\label{dontcare_padding}
	Every \HD solution which allows don't care symbols (eg. the algorithms from \Cref{sigman2d} and \Cref{approx2d}) can be extended to also calculate the Hamming distance for occurrences of $P$ which are not entirely contained in $T$.
	It can be done by padding the text with don't care symbols and it does not change the complexity of the solution.
\end{observation}


% MAIN ALGORITHM
\section{Proof of \Cref{main result}}
We show an algorithm which works in time $\tO(m^2 + mk^{5/4})$, assuming $2|n$ and $m < n \le \frac{3}{2}m$.
By \Cref{instancing}, our main result follows.

We start by running the algorithm from \Cref{approx2d} with $\varepsilon = 1$.
We construct the set $Q$ as the set of such vectors $q \in \Z^2$ for which the estimated value of $\Ham(P + q, T)$ is at most $2k$.
For every $q \in \set{0, \dots, n - m}^2 \setminus Q$ we say that $\Ham(P + q, T)$ equals $\infty$.
The next step is to calculate the exact value of $\Ham(P + q, T)$ for every $q \in Q$.

Let us consider the case when $|Q| \le 2m + m^2/k$.
We can run the algorithm from \Cref{kangaroos} and by the fact that $\Ham(P + q, T) \le 4k$ for every $q \in Q$, it will perform $\tO(m^2 + mk)$ operations.
We are left with the case when $|Q| > 2m + m^2/k$, in which we take advantage of the fact that some strings $P + q$ for $q \in Q$ must have a large overlap and small Hamming distance from each other, and thus $P$ must be periodic.


\newcommand{\T}{\mathcal{T}}
\renewcommand{\S}{\mathcal{S}}
\renewcommand{\P}{\mathcal{P}}
\newcommand{\U}{\mathcal{U}}
\newcommand{\V}{\mathcal{V}}
\newcommand{\F}{\mathcal{F}}
\renewcommand{\L}{\mathcal{L}}


% 2D PERIODICITY
\subsection{Two-dimensional periodicity} \label{periodicity_section}
In this section we introduce a range of new tools related to two-dimensional periodicity.
We then select some special periods of the pattern and show how to decompose it into some regularly structured monochromatic strings.


\begin{definition}[Periodicity]
	Consider any vector $\delta \in \Z^2$.
	We say that a string $S$ has an $\ell$-period $\delta$ when
	$$ \Ham(S + \delta, S) \le \ell. $$
\end{definition}


\begin{lemma} \label{periodicity_lemma}
	For every $u, v \in Q$, the vector $u - v$ is an $8k$-period of $P$.
	\begin{proof}
		$\Ham(P + u - v, P) = \Ham(P + u, P + v) \le \Ham(P + u, T) + \Ham(P + v, T) \le 4k + 4k. $
	\end{proof}
\end{lemma}


\begin{theorem} \label{get_periods}
	For a given $\ell \in \Z^+$ and a set of points $U \subseteq [\ell + 1]^2 $ such that $|U| > 4\ell$ there exist $s, t, s', t' \in U$ such that the following conditions hold for $w = t - s$ and $w' = t' - s'$:
	\begin{itemize}
		\item $0 < |w||w'| \le 22\frac{\ell^2}{|U|}$,
		\item $|\sin \alpha| \ge \frac{1}{2}$ where $\alpha$ is the angle between $w$ and $w'$,
		\item $w, w', -w, -w'$ are all contained in different quadrants.
	\end{itemize}
	Such $w, w'$ can be found in $\tO(|U|)$ operations.
	\begin{proof} See \Cref{get_periods_proof}. \end{proof}
\end{theorem}


We run the algorithm from \Cref{get_periods} on the set $Q$ (where $\ell = n - m \le m / 2$, thus $|Q| > 2m + m^2/k \ge 4\ell$).
We obtain vectors $\phi \in \Q_4$ and $\psi \in \Q_1$ which by \Cref{periodicity_lemma} are $\O(k)$-periods of $P$.
We will refer to those vectors throughout the rest of the description and we define $p = \phi \times \psi$.
Note that because $|Q| > 2m + m^2 / k$, we have $p \le |\phi||\psi| = \O(\min\set{m, k})$.


\begin{definition}[Lattice congruency]\label{lattice_congruency}
	We define $\L = \set{s\phi + t\psi : s, t \in \Z}$.
	We say that two vectors $u, v \in \Z^2$ are \textbf{lattice-congruent} and denote $u \equiv v$ when $u - v \in \L$ [Galil citation].
\end{definition}


\begin{lemma} \label{lattice_base}
	There exists a set of points $\Gamma \subseteq \Z^2$ such that $|\Gamma| = p$ and every point $u \in \Z^2$ is lattice-congruent to exactly one point $\gamma \in \Gamma$.
	\begin{proof} TODO \end{proof}
\end{lemma}


\begin{definition}[Parquet]\label{parquet_definition}
	We call a non-empty set $U \subseteq \Z^2$ a \textbf{parquet} when there exist some values $x_0, x_1, y_0, y_1, \phi_0, \phi_1, \psi_0, \psi_1 \in \Z$, which we will call its \textbf{signature}, such that
	$$ U = [x_0, x_1) \times [y_0, y_1) \cap \set{u : u \in \Z^2, \h{u} \in [\phi_0, \phi_1), \s{u} \in [\psi_0, \psi_1)}. $$
	\begin{enumerate}[a)]
		\item If additionally $x_1 - x_0 \ge |\x{\phi}| + |\x{\psi}|$ and $y_1 - y_0 \ge |\y{\phi}| + |\y{\psi}|$, then $U$ is a \textbf{spacious} parquet.
		\item If additionally $x_0, y_0 = -\infty$ and $x_1, y_1 = +\infty$, then $U$ is a \textbf{simple} parquet.
	\end{enumerate}
	Note that every simple parquet is spacious.
\end{definition}


\begin{definition}[Subparquet]\label{subparquet_definition}
	We call a non-empty set $V \subseteq \Z^2$ a \textbf{subparquet} when there exists a parquet $U$ and a point $\gamma \in \Z^2$ such that
	$$ V = \set{u : u \in U, u \equiv \gamma}.$$
	We call $V$ a spacious/simple subparquet when there exists $U$ which is (correspondingly) a spacious/simple parquet.
	We say that $V$ is lattice-congruent to some $v \in \Z^2$ (denoted as $V \equiv v$) when $v \equiv \gamma$.
	We similarly define the lattice congruency between two subparquets.
\end{definition}


\begin{theorem}[Subparquet convolution]\label{subparquet_convolution}
	For a given list of signatures of simple subparquets $U_0, \dots, U_{\ell - 1}$, list of signatures of subparquets $V_0, \dots, V_{\ell - 1}$ and a set of vectors $Q$ we can calculate
	$$ \sum_{i = 0}^{\ell - 1} |(U_i + q) \cap V_i| $$
	for every $q \in Q$ in total time $\tO(m^2 + \ell + |Q|)$, assuming that the subparquets only contain vectors of length $\O(m)$.
	\begin{proof} See \Cref{subparquet_convolution_proof}. \end{proof}
\end{theorem}


\begin{definition}[Parquet string]
	We call a string $S$ a spacious/simple (sub-)parquet string when $\d{S}$ is a spacious/simple (sub-)parquet.
\end{definition}


\begin{theorem}[Periodic string decomposition]\label{parquet_decomposition}
	A given spacious/simple parquet string $R$ with $\O(k)$-periods $\phi$ and $\psi$ can be partitioned in time $\tO(|\d{R}|)$ into $\O(k)$ monochromatic spacious/simple subparquet strings, correspondingly.
	\begin{proof} See \Cref{parquet_decomposition_proof}. \end{proof}
\end{theorem}


Since $|\x{\phi}|, |\y{\phi}|, |\x{\psi}|, |\y{\psi}| \le n - m \le m / 2$, the $m \times m$ string $P$ is a spacious parquet string and satisfies the assumptions of \Cref{parquet_decomposition}.
We partition $P$ into a set of strings $\V$.
We then group the strings based on the single character they contain.
Specifically, we construct $\V_\sigma = \set{V : V \in \V, \getchar{V} = \sigma}$ for every character $\sigma \in \Sigma$ present in $P$.


\begin{theorem}\label{sparse_algo}
	For a given set of monochromatic simple subparquet strings $\S$ we can calculate
	$$ \sum_{S \in \S} \Ham(P + q, S) $$
	for every $q \in Q$ in total time $\tO(m^2 + \sum_{S \in \S} |\V_{\getchar{S}}|)$, assuming that the sets $\d{S}$ for $S \in \S$ are some pairwise disjoint subsets of $\d{T}$.
	\begin{proof}
		Let $U = \bigcup_{S \in \S} \d{S}$. Observe that
		$$ \sum_{S \in \S} \Ham(P + q, S) = |(P + q) \cap U| - \sum_{S \in \S}\sum_{V \in \V_{\getchar{S}}} |\d{V + q} \cap \d{S}|.$$
		We can calculate $|(P + q) \cap U|$ for every $q \in Q$ with a single instance of FFT (see \Cref{sigman2d}) or by using prefix sums in time $\tO(m^2)$.
		To calculate the values
		$$ \sum_{S \in \S}\sum_{V \in \V_{\getchar{S}}} |\d{V + q} \cap \d{S}| $$
		we use the algorithm from \Cref{subparquet_convolution} (where $\ell = \sum_{S \in \S} |\V_{\getchar{S}}|$).
	\end{proof}
\end{theorem}


% TEXT DECOMPOSITION
\subsection{Text decomposition}
Because the text is not necessarily periodic, we unfortunately cannot use the same approach as for the pattern.
In this section we show how to decompose $T$ using a similar, but more nuanced approach.


\newcommand{\Ta}{T_\mathbf{a}}
\begin{definition}[Active text]
	We define the \textbf{active text} $\Ta$ as the restriction of $T$ to $\bigcup_{q \in Q} \d{P} + q$.
	For a point $u \in \Z^2$ we define its \textbf{border distance} as
	$\min\set{|u - v| : v \in \Z^2 \setminus \d{\Ta}} $,
	which we will denote as $\BD(u)$.
	For a set of points $U \subseteq \Z^2$, we define $\BD(U) = \max\set{\BD(u) : u \in U}$ and for a string $S$, we define $\BD(S) = \BD(\d{S})$.
	Note that we consider the maximum distance, not minimum.
\end{definition}


\begin{observation}
	$\Ham(P + q, T) = \Ham(P + q, \Ta)$ for every $q \in Q$.
\end{observation}


\begin{theorem}[Active text decomposition]\label{text_decomposition}
	For a given positive integer $\ell \le m$, we can partition the active text in time $\tO(m^2 + \ell k)$ into a set of $\O(\ell k)$ monochromatic simple subparquet strings
	and a string $F$ such that $\BD(F) = \O(m / \ell)$.
	\begin{proof} See \Cref{text_decomposition_proof}. \end{proof}
\end{theorem}


We partition $\Ta$ using the algorithm from \Cref{text_decomposition} with $\ell = mk^{-3/4}$ into a set of simple subparquet strings $\S$ and a string $F$,
where $|\S| = \O(mk^{1/4})$ and $\BD(F) = \O(k^{3 / 4})$.
For every $q \in Q$ we then have
$$ \Ham(P + q, \Ta) = \Ham(P + q, F) + \sum_{S \in \S} \Ham(P + q, S).$$

By \Cref{sparse_algo}, we can calculate $\sum_{S \in \S} \Ham(P + q, S)$ for every $q \in Q$ in time $\tO(m^2 + mk^{5/4})$,
since $\sum_{S \in \S} |\V_{\getchar{S}}| \le |\S| |\V| = \O(mk^{5/4})$.
In the next section we will introduce \Cref{dense_algo}, which states that we can calculate $\Ham(P + q, F)$ for every $q \in Q$ in total time $\tO(m^2 + mk^{1/2} \BD(F))$.
By substituting $\BD(F) = \O(k^{3/4})$, we get the complexity of $\tO(m^2 + mk^{5/4})$, which ends the main proof.


% BORDER PROXIMITY
\subsection{Border proximity}
In this section we explore the properties of strings defined only for the points close to the border.
We consider any non-empty string $S$, such that $\d{S} \subseteq \d{\Ta}$ and let $d = \BD(S)$.
We define a partitioning of $S$ into strings $S_1, \dots, S_4$, by splitting it through the middle with a horizontal and vertical line. 
Specifically
\begin{itemize}
	\item $S_1$ is the restriction of $S$ to $\set{n / 2, \dots, n - 1} \times \set{n / 2, \dots, n - 1}$ (upper right quarter),
	\item $S_2$ is the restriction of $S$ to $\set{0, \dots, n / 2 - 1} \times \set{n / 2, \dots, n - 1}$ (upper left quarter),
	\item $S_3$ is the restriction of $S$ to $\set{0, \dots, n / 2 - 1} \times \set{0, \dots, n / 2 - 1}$ (lower left quarter),
	\item $S_4$ is the restriction of $S$ to $\set{n / 2, \dots, n - 1} \times \set{0, \dots, n / 2 - 1}$ (lower right quarter).
\end{itemize}
We will demonstrate some characteristics of $S_1$, and by symmetry, generalize them to $S$.

\begin{lemma} \label{border_lemma}
	Assuming $d \le m/4$, there does not exist $u \in S_1$ and $v \in T_a$ such that $\x{v} - \x{u} \ge d$ and $\y{v} - \y{u} \ge d$.
	\begin{proof}
		Assume the contrary.
		Since $u \in S_1$, we have $\BD(u) \le d$, so there exists $w \in \Z^2 \setminus \d{\Ta}$ such that
		$\x{u} - d \le \x{w} \le \x{u} + d$ and
		$\y{u} - d \le \y{w} \le \y{u} + d$.
		Since $v \in T_a$, there exists $q \in Q$ such that $v \in [m]^2 + q$.
		We have
		$$ \x{w} \ge \x{u} - d \ge n / 2 - m / 4 \ge n - m > \x{q} $$
		and
		$$ \x{w} \le \x{u} + d \le \x{v} \le \x{q} + m - 1. $$
		Similarly we can show that $\y{q} \le \y{w} \le \y{q} + m - 1$, and thus $w \in [m]^2 + q$.
		Since $[m]^2 + q \subseteq \d{\Ta}$ and $w \not \in \Ta$, we get a contradiction.
	\end{proof}
\end{lemma}

\begin{theorem}\label{sigma_border}
	We can calculate $\Ham(P + q, S)$ for every $q \in Q$ in total time $\tO(m^2 + md |\Sigma|)$, where $|\Sigma|$ is the number of different characters present in both $P$ and $S$.
	\begin{proof} See \Cref{sigma_border_proof}. \end{proof}
\end{theorem}

\begin{theorem}\label{dense_algo}
	We can calculate $\Ham(P + q, S)$ for every $q \in Q$ in total time $\tO(m^2 + mdk^{1/2})$.
	\begin{proof} See \Cref{dense_algo_proof}. \end{proof}
\end{theorem}


\subsubsection{Proof of \Cref{sigma_border}} \label{sigma_border_proof}

We base our approach on the simple method of calculating the Hamming distance by running an instance of FFT for each unique character.
We will again utilize partitioning to reduce the problem to some smaller ones and then solve them naively.

\begin{definition}(width \& height)\label{width_and_height_definition}
	For a non-empty set $U \subseteq \Z^2$ we define its \textbf{width} as $\max\set{\x{u} - \x{v} + 1 : u, v \in U}$
	and its \textbf{height} as $\max\set{\y{u} - \y{v} + 1 : u, v \in U}$.
	For a non-empty string $R$ we define the width and height as the width and height of $\d{R}$.
\end{definition}

\begin{theorem}\label{general_fft}
	Given two non-empty strings $P$ and $T$ of widths $w_P, w_T$ and heights $h_P, h_T$, we can calculate $\Ham(P + q, T)$ for every $q \in \Z^2$, for which the result is non-zero, in total time $\tO((|\Sigma| + 1)(w_P + w_T)(h_P + h_T))$, where $|\Sigma|$ denotes the number of different characters present in both $P$ and $T$.
	\begin{proof}
		We can prove it by slightly generalizing \Cref{sigman2d}, although following the same method, and utilizing \Cref{dontcare_padding}.
	\end{proof}
\end{theorem}

We will assume that $d \le m / 4$, since for $d = \Omega(m)$ we can, by \Cref{general_fft}, calculate the results in time $\tO(m^2 + m^2|\Sigma|)$, which is sufficient.

Recall that 
$\Ham(P + q, S) = \Ham(P + q, S_1) + \dots + \Ham(P + q, S_4)$.
We will only show how to calculate $\Ham(P + q, S_1)$ for every $q \in Q$, since the other cases are symmetric.

We will take advantage of the fact that the points close to the border can overlap only with a small subset of points from the pattern when considering the occurrences fully contained in the active text.
Specifically, let us consider a string $P_0$, defined as the restriction of $P$ to $[m - d]^2$ and a string $P_1$, defined as the restriction of $P$ to $\d{P} \setminus \d{P_0}$.
Since the strings $P_0$ and $P_1$ partition $P$, we have
$$ \Ham(P + q, S_1) = \Ham(P_0 + q, S_1) + \Ham(P_1 + q, S_1).$$

\begin{lemma}\label{border_hamming_reduction}
	$\d{P_0 + q} \cap \d{S_1} = \emptyset$ for every $q \in Q$.
	\begin{proof}
		Let us assume the contrary.
		Select any $q \in Q$ such that $\d{P_0 + q} \cap \d{S_1}$ 
		contains some point $u$ and consider the point $v = (\x{u} + d, \y{u} + d)$.
		Since $u \in [m - d]^2 + q$, we have $v \in [m]^2 + q \subseteq \d{\Ta}$, thus the points $u \in S_1$ and $v \in \Ta$ contradict \Cref{border_lemma}.
	\end{proof}
\end{lemma}

\begin{observation}\label{border_hamming_split}
	$P_1$ can be partitioned into two strings $P_2$ and $P_3$ such that the width of $P_2$ and the height of $P_3$ are equal to $d$.
\end{observation}

By \Cref{border_hamming_reduction}, $\Ham(P_0 + q, S_1) = 0$ for every $q \in Q$
and by \Cref{border_hamming_split} we then have 
$$\Ham(P + q, S_1) = \Ham(P_1 + q, S_1) = \Ham(P_2 + q, S_1) + \Ham(P_3 + q, S_1). $$
for some newly constructed strings $P_2$ and $P_3$, such that the width of $P_2$ and the height of $P_3$ equals $d$.
We calculate $\Ham(P_2 + q, S_1)$ and $\Ham(P_3 + q, S_1)$ for every $Q$ independently and sum the results.
We only show how to calculate $\Ham(P_2 + q, S_1)$, since the other case is symmetric.

We will now partition $S_1$.
Consider an array of strings $U_0, \dots, U_{\lceil n / d \rceil - 1}$, where $U_i$ is the restriction of $S_1$ to $\set{id, \dots, id + d - 1} \times [n] \cap \d{S_1}$.
For the sake of formality (since the maximum/minimum of an empty set is undefined), let $V_0, \dots, V_{\ell - 1}$ consist of all non-empty strings $U_i$, given in the increasing order of $i$.
Observe that $V_0, \dots, V_{\ell - 1}$ partition $S_1$ and their width is not greater than $d$.

For each $i \in [\ell]$ we find $h_i \in \Z^+$, which we define as the minimal number such that $(\x{u}, \y{u} + h_i) \not \in \Ta$ for every $u \in V_i$.

\begin{lemma}\label{sum_of_h}
	The sum of all $h_i$ is $\O(m)$.
	\begin{proof}
		Since for $\ell < 2$, the proof is trivial, we assume $\ell \ge 2$.
		For every $i$ (since $h_i$ is minimal) there exists a point $u_i \in V_i$, such that $(\x{u_i}, \y{u_i} + h_i - 1) \in \Ta$.
		It can be shown that for all $i \ge 2$ we have
		$$ h_{i} \le \y{u_{i - 2}} - \y{u_{i}} + d,$$
		since if that was not the case for some $i$, then the points $u_{i - 2}$ and $v = (\x{u_{i}}, \y{u_{i}} + h_{i} - 1)$ would contradict \Cref{border_lemma}.
		We can conclude that
		$$
			\sum_{i = 0}^{\ell - 1} h_i
			\le h_0 + h_1 + \sum_{i = 2}^{\ell - 1} (\y{u_{i - 2}} - \y{u_i} + d)
			= h_0 + h_1 + \y{u_0} + \y{u_1} - \y{u_{\ell - 2}} - \y{u_{\ell - 1}} + (\ell - 2)d 
			= \O(m).
		$$
	\end{proof}
\end{lemma}


\begin{observation}\label{area_bound}
	For every $i$, the height of $V_i$ is not greater than $h_i$.
	By \Cref{sum_of_h} we have $|S_1| = \O(md)$ and by extension $|S| = \O(md)$.
\end{observation}


For every $i \in [l]$ we construct the string $L_i$ as the restriction of $P_2$ to $[m] \times [m - h_i] \cap \d{P_2}$
and the string $H_i$ as the restriction of $P_2$ to $\d{P_2} \setminus \d{L_i}$.
Since $L_i$ and $H_i$ partition $V_i$, we have
$$ \Ham(P_2 + q, S_1) = \sum_{i = 0}^{\ell - 1} \Ham(P_2 + q, V_i) = \sum_{i = 0}^{\ell - 1} \Ham(L_i + q, V_i) + \sum_{i = 0}^{\ell - 1} \Ham(H_i + q, V_i). $$

\begin{lemma}\label{pattern_height_reduction}
	$\d{L_i + q} \cap \d{V_i} = \emptyset$ for every $q \in Q$ and $i \in [\ell]$.
	\begin{proof}
		Let us assume the contrary.
		Select any $q \in Q$ and $i \in [\ell]$, such that $\d{L_i + q} \cap \d{V_i}$ 
		contains some point $u$ and consider the point $v = (\x{u}, \y{u} + h_i)$.
		Since $u \in [m] \times [m - h_i] + q$, we have $v \in [m]^2 + q \subseteq \d{\Ta}$, thus $v \in \Ta$, which contradicts the definition of $h_i$.
	\end{proof}
\end{lemma}

By \Cref{pattern_height_reduction}, for every $q \in Q$ we have $\sum_{i = 0}^{\ell - 1} \Ham(L_i + q, V_i) = 0$, thus our result is equal to $\sum_{i = 0}^{\ell - 1} \Ham(H_i + q, V_i)$.
We run the algorithm from \Cref{general_fft} for every pair of $H_i$ and $V_i$ and, since both $H_i$ and $V_i$ have widths not greater than $d$ and heights not greater than $h_i$, we obtain the total complexity of $\tO(\sum_{i = 0}^{\ell - 1} (|\Sigma| + 1)dh_i)$, 
which, by \Cref{sum_of_h}, is $\tO(m^2 + md|\Sigma|)$.

\subsubsection{Proof of \Cref{dense_algo}} \label{dense_algo_proof}

Recall the construction of the sets $\V_\sigma$ described in \Cref{periodicity_section}.
We define $\sigma \in \Sigma$ to be a \textbf{frequent} character if $|\V_\sigma| \ge \sqrt{k}$ and if $|\V_\sigma| < \sqrt{k}$, we call it an \textbf{infrequent} character.
\begin{observation}\label{frequent_character_bound}
	The number of different frequent characters is $\O(\sqrt{k})$.
\end{observation}


We partition $S$ into two strings $F$ and $I$, based on character frequency,
so that $F$ consists of only the frequent characters and $I$ consists of only the infrequent ones.
For every $q \in Q$ we then have 
$$\Ham(P + q, S) = \Ham(P + q, F) + \Ham(P + q, I).$$

By \Cref{frequent_character_bound} and \Cref{sigma_border}, we can calculate $\Ham(P + q, F)$ for every $q \in Q$ in total time $\tO(m^2 + mdk^{1/2})$, since $\BD(F) \le d$. 

We partition $I$ into $|\d{I}|$ strings, one per every $u \in I$.
Specifically, let $I_u$ be the restriction of $I$ to $\set{u}$ for every $u \in I$.
We have $\Ham(P + q, I) = \sum_{u \in I} \Ham(P + q, I_u)$ for every $q \in Q$.
By \Cref{subparquet_definition}, $I_u$ are simple subarquet strings, and thus, we can by \Cref{sparse_algo} calculate the results in $\tO(m^2 + \sum_{u \in I} |\V_{I(u)}|)$.
Since $I(u)$ is an infrequent character for every $u \in I$, we have $|\V_{I(u)}| < k^{1/2}$ for every $u \in I$.
By \Cref{area_bound} we have $|\d{I}| = \O(md)$, and thus the total complexity is $\tO(m^2 + mdk^{1/2})$.

% PHI PSI FINDER
\subsection{Proof of \Cref{get_periods}} \label{get_periods_proof}
First, we find any closest pair of vectors $s, t \in U$ by running the standard $\tO(|U|)$ time algorithm and denote $w = t - s$.
We define a partial order $\le_{w}$ where $v \le_w u$ for some $u, v \in U$ when at least one condition holds:
\begin{enumerate}[(a)]
	\item $u = v$,
	\item $u - v$ and $w$ belong to the same quadrant,
	\item $\alpha \in (-\frac{\pi}{6}, \frac{\pi}{6})$ where $\alpha$ is the angle between $w$ and $u - v$.
\end{enumerate}
We find the longest chain $C$ and the longest antichain $A$ using dynamic programming in $\tO(|U|)$ operations.
We then find any closest pair of vectors $s', t' \in A$ and denote $w' = t' - s'$.
We have the following inequalities:
\begin{enumerate}[(i)]
	\item $|U| \le |C| |A|$ (by Dilworth's theorem),
	\item $(|C| - 1) |w| \le (1 + \sqrt{3})\ell$ (roughly by the fact that vectors in $C$ must be increasing in a certain direction), 
	\item $(|A| - 1) |w'| \le 2 \ell$ (by using a similar argument for vectors in $A$).
\end{enumerate}
By considering the assumption $|U| > 4\ell$ it can be proven that $|w||w'| \le 22 \frac{\ell^2}{|U|}$ and the other conditions also hold.


% SUBPARQUET CONVOLUTION
\newcommand{\W}{\mathcal{W}}
\subsection{Proof of \Cref{subparquet_convolution}} \label{subparquet_convolution_proof}
Throughout this section we will denote $D = \set{u : u \in \L, \h{u} \ge 0, \s{u} \ge 0}$, where $\L$ is the set introduced in \Cref{lattice_congruency}.

\begin{lemma}\label{primitive_conv}
	Given a set of subparquets $\V$ and a set of points $Q$, we can calculate
	$$ \sum_{V \in \V} |(D + q) \cap V| $$
	for every $q \in Q$ in total time $\tO(n^2 + |Q| + |\V|)$, assuming that every $V \in \V$ consists of vectors of length $\O(n)$.
	\begin{proof}
		For every $u \in \Z^2$ let us define $\score(u) = |\set{V : V \in \V, u \in V}|$. Observe that
		$$ \sum_{V \in \V} |(D + q) \cap V| = \sum_{u \in D + q} \score(u). $$
		We start by explicitly calculating the scores.
		We find the maximum length of a vector that some $V \in \V$ is defined for, which we denote $\ell$.
		We construct the set $U \subseteq \Z^2$ of all vectors of length at most $\ell$.
		By the assumption, we have $\ell = \O(n)$, and thus $|U| = \O(l ^ 2) = \O(n ^ 2)$.
		We observe that since all the scores are zero for points outside of $U$, we can only calculate them for $u \in U$.
		
		We find the set $\Gamma$ introduced in \Cref{lattice_base} and for every $\gamma \in \Gamma$ we construct $U_\gamma = U \cap (\L + \gamma)$.
		Consider any $u \in U_\gamma$ for some fixed $\gamma \in \Gamma$ and any $V \in \V$.
		We observe that if $V \not \equiv \gamma$, then $u \not \in V$ and thus $V$ does not contribute to $\score(u)$.
		If $V \equiv \gamma$, then we can find a parquet $W$ such that $V = W \cap (\L + \gamma)$ and we have
		$u \in V \Leftrightarrow u \in W \cap (\L + \gamma) \Leftrightarrow u \in W$.
		Thus, if we denote $\W_\gamma$ as the set of parquets $W$ obtained for every $V \in \V$ such that $V \equiv \gamma$, then $\score(u)$ for $u \in U_\gamma$ is the number of parquets $W \in \W_\gamma$ such that $u \in W$.
		We calculate $\score(u)$ for every $u \in U_\gamma$ by sweeping $U_\gamma$ and $\W_\gamma$ in time $\tO(|U_\gamma| + |\W_\gamma|)$.
		We do it independently for every $\gamma \in \Gamma$, performing $\tO(|U| + |\V|) = \tO(n^2 + |\V|)$ operations in total.

		Now consider a query vector $q \in Q$.
		Let $\gamma \in \Gamma$ be such that $q \equiv \gamma$.
		We have already showed that the sum of scores for $u \in D + q$ is equal to the sum of scores for $u \in (D + q) \cap U$.
		Since $(D + q) \cap U = (D + q) \cap U_\gamma$, we see that the result is the sum of scores for such $u \in U_\gamma$, for which $\h{u} \ge \h{q}$ and $\s{u} \ge \s{q}$.
		If we denote $Q_\gamma = Q \cap (\L + \gamma)$, we see that we can calculate the results for all $q \in Q_\gamma$ by sweeping $Q_\gamma$ and $U_\gamma$ in time $\tO(|Q_\gamma| + |U_\gamma|)$.
		We do it independently for every $\gamma \in \Gamma$, performing $\tO(|Q| + |U|) = \tO(n^2 + |Q|)$ operations in total.
	\end{proof}
\end{lemma}

\begin{lemma}\label{primitive}
	For any simple subparquet $U$ we can find $w_0, \dots, w_3 \in \Z^2$ such that
	$$ |U \cap X| = \sum_{j = 0}^3 (-1)^j |(D + w_j) \cap X|$$
	for every $X \subseteq \Z^2$.
	If $U$ consists of vectors of length $\O(n)$, then $w_0, \dots, w_3$ are of length $\O(n)$.
	\begin{proof}
		TODO
	\end{proof}
\end{lemma}

We apply \Cref{primitive} to every $U_i$ and find $w_{i, 0}, \dots, w_{i, 3}$ so that we have
\eq{
	\sum_{i = 0}^{\ell - 1}|(U_i + q) \cap V_i| 
= \sum_{i = 0}^{\ell - 1}|U_i \cap (V_i - q)| 
= \sum_{i = 0}^{\ell - 1} \sum_{j = 0}^3 (-1)^j |(D + w_{i, j}) \cap (V_i - q)| = \\
= \sum_{j = 0}^3 (-1)^j \sum_{i = 0}^{\ell - 1} |(D + q) \cap (V_i - w_{i, j})|.
}
By \Cref{primitive_conv} we can independently calculate the values $\sum_{i = 0}^{\ell - 1} |(D + q) \cap (V_i - w_{i, j})|$ for every $j$ by running the algorithm for $\V_j = \set{V_i - w_{i, j} : i \in [\ell]}$ and $Q$.


% PARQUET DECOMPOSITION
\subsection{Proof of \Cref{parquet_decomposition}} \label{parquet_decomposition_proof}


\begin{definition}[Lattice graph]
	For a set $U \subseteq \Z^2$ we define its \textbf{lattice graph} $G_U = (U, E_U)$ where
	$$ E_U = \bigset{\set{u, u + \delta} : \delta \in \set{\phi, \psi}, u \in U, u + \delta \in U} $$ 
	so every vector is connected with its translations by $\phi, \psi, -\phi, -\psi$.
\end{definition}


\begin{lemma}
	If $U$ is a spacious subparquet, then $G_U$ is connected.
\end{lemma}


Firstly, we partition $R$ into a set of subparquet strings $\S$.
For every non-empty $S \in \S$ we consider a lattice graph $G_{\d{S}}$. If $S$ is not monochromatic, then since $G_{\d{S}}$ is connected, there must exist a pair of neighboring vectors $v, w$ such that $S(v) \neq S(w)$.
We select any such pair and partition $S$ into spacious (or simple if $S$ is simple) subparquet strings $S'$ and $S''$ such that $v \in S'$ and $w \in S''$.
For example if $v = w + \phi$, then $S' = \set{u : u \in S, \s{u} \le \s{v}}$ and $S'' = \set{u : u \in S, \s{u} > \s{v}}$.
In the cases when $v = w + \delta$ for $\delta \in \set{-\phi, \psi, -\psi}$ the construction in similar.

We can recursively partition $S'$ and $S''$ further until we obtain monochromatic strings.
Because $R$ has $\O(k)$-periods $\phi$ and $\psi$, the total number of neighbor pairs $v, w$ such that $S(v) \neq S(w)$ is $\O(k)$ throughout all $S \in \S$.
Thus the total number of recursive calls is $\O(k)$ and because $|\S| = \O(k)$, the total number of constructed strings is $\O(k)$.
The algorithm can be implemented to work in time $\tO(|\d{R}|)$.


% TEXT DECOMPOSITION
\subsection{Proof of \Cref{text_decomposition}} \label{text_decomposition_proof}

In this section we will consider an empty set to be a valid parallelogram.

\begin{definition}
	For a set of points $U \subseteq \R^2$ we will denote
	$$ \X(U) = \set{\x{u} : u \in U}, \quad \Y(U) = \set{\y{u} : u \in U}.$$
\end{definition}

\newcommand{\IQ}{\mathbb{R} \setminus \mathbb{Q}}
\begin{observation}\label{line_existence}
	For any given $\ell \in \Z^+$ and $v \in \Z^2$ we can find an array of parallel lines $f_0, f_1, \dots, f_\ell$, where $f_i = \set{u : u \in \R^2, v \times u = c_i}$ for some $c_i \in \IQ$, such that
	\begin{itemize}
		\item $c_0 < v \times u < c_\ell$ for every $u \in [n]^2$, or namely, the set $[n]^2$ is between $f_0$ and $f_\ell$,
		\item $0 < c_{i + 1} - c_i = \O(n|v| / \ell)$ for every $i \in [\ell]$, or namely, the distance between every two consecutive lines is $\O(n / \ell)$.
	\end{itemize}
\end{observation}

We use \Cref{line_existence} with $v = \phi$ to construct the lines $h_0, \dots, h_\ell$ and with $v = \psi$ to construct the lines $s_0, \dots, s_\ell$.
For every $i, j \in [\ell + 1]$ we construct a point $w_{i, j}$ as the intersection of $h_i$ and $s_j$ (it is easy to see that since $\phi$ and $\psi$ are not collinear, $h_i$ and $s_j$ are not parallel).
For every $i, j \in [\ell]$ we construct a parallelogram $p_{i, j}$ defined as the area between $s_i$ and $s_{i + 1}$ intersected with the area between $h_j$ and $h_{j + 1}$.
Specifically,
$$p_{i, j} = \set{u : u \in \R^2, \h{u} \in [\h{w_{i, j}}, \h{w_{i + 1, j + 1}}], \s{u} \in [\s{w_{i, j}}, \s{w_{i + 1, j + 1}}]}.$$
For a better reference, the vertices of $p_{i, j}$ are $w_{i, j}, w_{i + 1, j}, w_{i + 1, j + 1}, w_{i, j + 1}$.
Observe that every $u \in [n]^2$ is contained strictly inside exactly one parallelogram $p_{i, j}$.

\begin{lemma}\label{monotonicity_lemma}
	For every $i \in [\ell - 1]$ and $j \in [\ell]$ we have
	\eq{
	\min \X(p_{i, j}) < \min \X(p_{i + 1, j}), \quad
	&\min \Y(p_{i, j}) \le \min \Y(p_{i + 1, j}), \\
	\max \X(p_{i, j}) < \max \X(p_{i + 1, j}), \quad
	&\max \Y(p_{i, j}) \le \max \Y(p_{i + 1, j})
	}
	and for every $i \in [\ell]$ and $j \in [\ell - 1]$ we have
	\eq{
		\min \X(p_{i, j}) \ge \min \X(p_{i, j + 1}), \quad
		&\min \Y(p_{i, j}) < \min \Y(p_{i, j + 1}), \\
		\max \X(p_{i, j}) \ge \max \X(p_{i, j + 1}), \quad
		&\max \Y(p_{i, j}) < \max \Y(p_{i, j + 1}).
	}
	\begin{proof}
		It follows from the fact that we selected $\phi \in [0, +\infty) \times (-\infty, 0)$ and $\psi \in (0, +\infty) \times [0, +\infty)$.
		For example, to prove the first inequality, we can consider a point $u \in p_{i + 1, j}$, such that $\x{u} = \min \X(p_{i + 1, j})$
		and then construct a point $v \in p_{i, j}$, such that $v = u - t\psi$ for some $t > 0$, and thus $\min \X(p_{i, j}) \le \x{v} \le \x{u} = \min \X(p_{i + 1, j})$.
		The other inequalities can be proven analogously.
	\end{proof}
\end{lemma}

\begin{lemma}\label{distance_bound_lemma}
	For every $i, j \in [\ell]$ and every $u, v \in \X(p_{i, j}) \times \Y(p_{i, j})$, we have $|u - v| = \O(n / \ell)$.
	\begin{proof} See \Cref{distance_bound_lemma_proof}. \end{proof}
\end{lemma}

Consider the case when $\max \X(p_{i, j}) - \min \X(p_{i, j}) \ge m / 4$ for some $i, j \in [\ell]$.
By \Cref{distance_bound_lemma}, we would have $m / 4 \le \max \X_(p_{i, j}) - \min \X(p_{i, j}) = \O(n / \ell)$, and thus $\ell = \O(1)$.
In that case we can return a trivial partitioning where $F = \Ta$ and the set of monochromatic strings is empty, since $\BD(\Ta) = \O(m)$.
We can use the same argument if we have $\max \Y(p_{i, j}) - \min \Y(p_{i, j}) \ge m / 4$ for some $i, j \in [\ell]$.
Thus, from now on we will assume that $\max X_{i, j} - \min X_{i, j} < m / 4$ and $\max Y_{i, j} - \min Y_{i, j} < m / 4$ for every $i, j \in [\ell]$. 

Let $z = \frac{n - 1}{2}$. 
We split the plain with two lines $x = z$ and $y = z$ into four quarters:
\begin{enumerate}[1)]
	\item $K_1 = (z, +\infty) \times (z, +\infty)$,
	\item $K_2 = (-\infty, z) \times (z, +\infty)$,
	\item $K_3 = (-\infty, z) \times (-\infty, z)$,
	\item $K_4 = (z, +\infty) \times (-\infty, z)$.
\end{enumerate}
\newcommand{\I}{\mathcal{I}}
\newcommand{\G}{\mathcal{G}}
\newcommand{\C}{\mathcal{C}}
Let us denote by $\I$ the set of all parallelograms $p_{i, j}$, such that they intersect with the line $x = z$ or with the line $y = z$ (or both).
Observe that every parallelogram $p_{i, j} \not \in \I$ must be fully contained in one of the quarters, meaning $p_{i, j} \subseteq K_d$ for some $d \in \set{1, \dots, 4}$.

\begin{lemma}\label{I_size_bound}
	$|\I| = \O(\ell).$
	\begin{proof}
		Consider the line $x = z$, denoted $f$.
		It intersects with every line $h_0, \dots, h_\ell$ at most once (and does not overlap with any of them).
		Similarly, it intersects with every line $s_0, \dots, s_\ell$ at most once.
		Denote the set of such intersections as $U$.
		For every parallelogram $p \in \I$, there must exist $u \in U$, such that $u \in p$.
		For every $u \in U$, there are at most four parallelograms $p \in \I$, such that $u \in p$,
		thus $|\I| \le 4|U| = \O(\ell)$.
	\end{proof}
\end{lemma}

Now consider any $j \in [\ell]$.
By \Cref{monotonicity_lemma}, we can find $s, t \in [\ell + 1]$, such that the array $p_{0, j}, \dots, p_{\ell - 1, j}$ is split into three groups:
\begin{enumerate}[a)]
	\item $p_{0, j}, \dots, p_{s - 1, j}$, which includes only parallelograms fully contained in $K_3$, \label{full in 3}
	\item $p_{s, j}, \dots, p_{t - 1, j}$, which does not include any parallelogram fully contained in $K_1$ or $K_3$,
	\item $p_{t, j}, \dots, p_{\ell - 1, j}$, which includes only parallelograms fully contained in $K_1$. \label{full in 1}
\end{enumerate}
We now ''merge together'' the parallelograms from group (\ref{full in 3}) and from group (\ref{full in 1}).
Specifically, we construct
$$
g^3_j = \bigcup_{i = 0}^{s - 1} p_{i, j}, \quad 
g^1_j = \bigcup_{i = t}^{\ell - 1} p_{i, j}.
$$
We do it for every $j \in [\ell]$.
Observe that the sets $g^1_0, \dots, g^1_{\ell - 1}$ are are parallelograms (possibly empty) and that they cover the same area as all the fully contained in $K_1$ parallelograms $p_{i, j}$.
The same is true for $g^3_0, \dots, g^3_{\ell - 1}$ and the parallelograms in $K_3$. 

Now for every $i \in [\ell]$ we similarly find $s, t \in [\ell + 1]$, such that the array $p_{i, 0}, \dots, p_{i, \ell - 1}$ is split into three groups:
\begin{enumerate}[a)]
	\item $p_{i, 0}, \dots, p_{i, s - 1}$, which includes only parallelograms fully contained in $K_4$,
	\item $p_{i, s}, \dots, p_{i, t - 1}$, which does not include any parallelogram fully contained in $K_2$ or $K_4$,
	\item $p_{i, t}, \dots, p_{i, \ell - 1}$, which includes only parallelograms fully contained in $K_2$,
\end{enumerate}
and then construct
$$
g^4_i = \bigcup_{j = 0}^{s - 1} p_{i, j}, \quad 
g^2_i = \bigcup_{j = t}^{\ell - 1} p_{i, j}.
$$
We denote
$$
\G = \set{g^1_0, \dots, g^1_{\ell - 1}}  
\cup \set{g^2_0, \dots, g^2_{\ell - 1}}
\cup \set{g^3_0, \dots, g^3_{\ell - 1}}
\cup \set{g^4_0, \dots, g^4_{\ell - 1}}.
$$
Again observe that for every $u \in [n]^2$ there exists exactly one parallelogram $p \in \G \cup \I$, such that $u \in p$, and since the sides of $p$ do not contain integer points, $u$ lays strictly inside $p$.

\begin{definition}[Coverability]
	We say that a set $U \subseteq \Z^2$ is \textbf{coverable} if $U \subseteq \d{P + q}$ for some $q \in Q$.
\end{definition}

\begin{lemma}
	For every $p \in \I$, the set $p \cap \Z^2$ is coverable.
	\begin{proof}
%		By \Cref{type 5 lemma}, and since the other cases are symmetric, we can assume that there exists a point $v \in U^\R_{i, j}$, such that $\x{v} = n / 2$ and $\y{v} \ge n / 2$.
%		Thus we have $\min X_{i, j} \ge n / 2 - m / 4$ and $\min Y_{i, j} \ge n / 2 - m / 4$.
%		Consider a point
%		$$ p = (\max \set{\x{u} : u \in U_{i, j}}, \max \set{\y{u} : u \in U_{i, j}}).$$
%		Since $U_{i, j}$ is not of type \Rom{6}, $p \in \Ta$ and there exists $q \in Q$, such that $p \in [m]^2 + q$, thus $\x{u} \le \x{q} + m - 1$ and $\y{u} \le \y{q} + m - 1$ for all $u \in U_{i, j}$.
%		We have 
%		$$\x{q} \le n - m \le n / 2 - m / 4 \le \min X_{i, j} \le \x{u}$$
%		for all $u \in U_{i, j}$ and we can silimarly get $\y{q} \le \y{u}$, thus $U_{i, j} \subseteq [m]^2 + q$.
	[FINISH]
	\end{proof}
\end{lemma}

\begin{lemma}\label{parallelogram_split_lemma}
	For every $g \in \G$ we can construct two parallelograms $c$ and $b$, such that
	\begin{itemize}
		\item $c \cap \Z^2$ is coverable,
		\item $b \cap \Z^2$ is a $\O(n / \ell)$-periphery,
		\item $g \cap \Z^2$ is partitioned into $b \cap \Z^2$ and $c \cap \Z^2$.
	\end{itemize}
	\begin{proof} See \Cref{parallelogram_split_lemma_proof}. \end{proof}
\end{lemma}

\begin{lemma}\label{coverable is periodic}
	The restriction of $T$ to a coverable set has $\O(k)$-periods $\phi$ and $\psi$.
	\begin{proof}
		Let $R$ denote the restriction. For $q \in Q$, such that $\d{R} \subseteq \d{P + q}$ we have
		\eq{
			\Ham(R + \phi, R) \le \Ham(R + \phi, P + q + \phi) + \Ham(P + q + \phi, P + q) + \Ham(P + q, R) \le \\
			\le \Ham(T, P + q) + \Ham(P + \phi, P) + \Ham(P + q, T) = \O(k)
		}
		and identically $\Ham(R + \psi, R) = \O(k)$.
	\end{proof}
\end{lemma}

[FINISH]
%We split every non-empty set from $\G$ (by \Cref{partition_lemma}) into a coverable simple parquet $S$ and a simple parquet $R$, where $\BD(R) = \O(n / \ell)$.
%We construct a family $\S$ of all the obtained sets $S$ and all sets $U_{i, j}$ of type 5.
%We also construct a family $\mathcal{R}$ of all the obtained sets $R$ intersected with $\d{\Ta}$.
%Observe that $\S \cup \mathcal{R}$ forms a partitioning of $\Ta$.
%By \Cref{5 is coverable} and \Cref{coverable is periodic}, we can see that every restriction of $T$ to some set from $\S$ is periodic.
%We can thus construct all such restrictions and use the algorithm from \Cref{parquet_decomposition} to partition each of them into $\O(|\S|k)$ monochromatic simple subparquets.
%Since $|\S| = \O(\ell)$, the total number of constructed subparquets is $\O(\ell k)$.
%Now construct the restriction of $T$ to $\bigcup_{R \in \mathcal{R}} R$, denoted $F$.
%We have $\BD(F) = \O(m / \ell)$.


% PROOF OF SPLIT LEMMA
\subsubsection{Proof of \Cref{parallelogram_split_lemma}} \label{parallelogram_split_lemma_proof}
For an empty parallelogram, the proof is trivial.
Consider a non-empty set $g^1_j$ for some $j \in [\ell]$.
We denote $B = K_1 \cap \Z^2 \setminus \d{\Ta}$.

\begin{lemma}\label{domination_lemma}
	There does not exist a pair of points $u \in \Ta$ and $v \in B$, such that $\x{u} \ge \x{v}$ and $\y{u} \ge \y{v}$. 
	\begin{proof}
	\end{proof}
\end{lemma}

\begin{lemma}\label{coverability_condition}
	Every set $U \subseteq K_1 \cap \Z^2$, such that $(\max \X(U), \max \Y(U)) \in \Ta$ is coverable.
	\begin{proof}
	\end{proof}
\end{lemma}

Consider a non-empty set $g^1_j$ for some $j \in [\ell]$.
Recall that there exists $t \in [\ell + 1]$, such that
$ g^1_j = \bigcup_{i = t}^{\ell - 1} p_{i, j}.$
We find 
$$ f = \min\set{i : i \in \set{t, \dots, \ell - 1}, (\floor{\max \X(p_{i, j})}, \floor{\max \Y(p_{i, j})}) \in B}.$$
If the minimum does not exist, we set $f = \ell$.
We then construct the parallelograms
$$c = \bigcup_{i = 0}^{f - 1} p_{i, j}, \quad b = \bigcup_{i = f}^{\ell - 1} p_{i, j}.$$

We now show that the set $c \cap \Z^2$ is coverable.
If $c \cap \Z^2$ is empty, then it is coverable, so let us assume it is not.
In that case $f > 0$.
It is clear that $c \cap \Z^2 \subseteq K_1$.
By \Cref{monotonicity_lemma}, we have 
\eq{
\max \X(c) = \max \X(p_{f - 1, j}), \\
\max \X(c) = \max \Y(p_{f - 1, j}),
}
and thus
\eq{
	\max \X(c \cap \Z^2) &\le \floor{\max \X(c)} = \floor{\max \X(p_{f - 1, j})}, \\
	\max \Y(c \cap \Z^2) &\le \floor{\max \Y(c)} = \floor{\max \Y(p_{f - 1, j})}.
}

We see that $(\max \X(c \cap \Z^2), \max \Y(c \cap \Z^2)) \in \Ta$, since it would otherwise contradict \Cref{domination_lemma},
considering that
$(\floor{\max \X(p_{f - 1, j})}, \floor{\max \Y(p_{f - 1, j})}) \in \Ta$.
We see that $c \cap \Z^2$ satisfies the conditions of \Cref{coverability_condition}, thus $c \cap \Z^2$ is coverable.

We now show that the set $b \cap \Z^2$ is a $\O(n / \ell)$-periphery.
If it is empty, then the proof is trivial, so let us assume it is not.
In that case $f < \ell$.
Denote $v = (\floor{\max \X(p_{f, j})}, \floor{\max \Y(p_{f, j})})$.
By definition, $v \in B$.
Consider any point $u \in b \cap \Z^2$.
There exists exactly one $i \in \set{f, \dots, \ell - 1}$, such that $u$ lays strictly inside $p_{i, j}$.
[FINISH]


% PROOF OF DISTANCE BOUND
\subsubsection{Proof of \Cref{distance_bound_lemma}} \label{distance_bound_lemma_proof}

Consider a parallelogram $p_{i, j}$ for some $i, j \in [\ell]$.

\begin{lemma}\label{distance_bound_aux}
	For every $u, v \in p_{i, j}$, we have $|u - v| = \O(n / \ell)$.
	\begin{proof}
		Consider any $u, v \in p_{i, j}$ and denote $w = u - v$.
		By definition of $p_{i, j}$, we have
		$$ |\h{w}| = |\h{(u - v)}| = |\h{u} - \h{v}| = \O(n|\phi| / \ell) $$
		and similarly $|\s{w}| = \O(n|\psi| / \ell)$.
		Since $\phi$ and $\psi$ are not collinear, there exist $s, t \in \mathbb{R}$, such that $w = s\phi + t\psi$.
		Recall that by \Cref{get_periods} we have $|\phi \times \psi| \ge \frac{1}{2}|\phi||\psi|$ (since $|\sin \alpha| \ge 1/2$), thus
		$$ \frac{1}{2}|t||\phi||\psi| \le |t||\phi \times \psi| = |\phi \times (s\phi + t\psi)| = |\h{w}| = \O(n|\phi| / \ell), $$
		which gives us $|t\phi| = \O(n / \ell)$.
		We can similarly prove that $|s\psi| = \O(n / \ell)$ and finally
		$$ |w| = |s\phi + t\psi| \le |s\phi| + |t\psi| = \O(n / \ell). $$
	\end{proof}
\end{lemma}

\begin{lemma}\label{distance_bound_aux2}
	For every point $u \in \X(p_{i, j}) \times \Y(p_{i, j})$ there exists a point $v \in p_{i, j}$, such that $|u - v| = \O(n / \ell)$.
	\begin{proof}
		There exists a point $w \in p_{i, j}$, such that $\x{u} = \x{w}$, and a point $v \in p_{i, j}$, such that $\y{u} = \y{v}$.
		By \Cref{distance_bound_aux}, we have
		$$|u - v| = |\x{u} - \x{v}| = |\x{w} - \x{v}| \le |w - v| = \O(n / \ell).$$ 
	\end{proof}
\end{lemma}

Consider any $u, v \in p_{i, j}$.
By \Cref{distance_bound_aux2}, there exist $u', v' \in p_{i, j}$ such that $|u - u'| = \O(n / \ell)$ and $|v - v'| = \O(n / \ell)$.
By \Cref{distance_bound_aux}, we have
$$ |u - v| \le |u - u'| + |u' - v'| + |v' - v| = \O(n / \ell). $$

\bibliographystyle{plainurl}
\bibliography{references}

\end{document}
