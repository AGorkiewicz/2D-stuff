\documentclass[11pt, letterpaper]{article}
\usepackage[left=1in,right=1in,top=1in,bottom=1in]{geometry}
\pdfoutput=1
\usepackage{amsthm,amssymb,amsmath,amsmath,amsfonts}  
\usepackage{xspace,enumerate}
\usepackage[dvipsnames]{xcolor}
\usepackage{thmtools}
\usepackage{thm-restate}
\usepackage[cmbtt]{bold-extra}
\usepackage[T1]{fontenc}
\usepackage{makecell}
\usepackage{listings}
\usepackage{multirow}
\usepackage{diagbox}
\usepackage[ruled,noline,noend]{algorithm2e}
\usepackage{tabularx}
\usepackage{epsfig}
\usepackage{verbatim}
\usepackage[noadjust]{cite}
\usepackage{authblk}
\usepackage{todonotes}

\usepackage[colorlinks=true,urlcolor=Blue,citecolor=Green,linkcolor=BrickRed]{hyperref}
\usepackage[capitalise]{cleveref}

\pagestyle{myheadings}

\title{Two-dimensional pattern matching with $k$ mismatches}
\author[1]{Jonas Ellert}
\author[2]{Paweł Gawrychowski}
\author[3]{Adam Górkiewicz}
\author[4]{Tatiana Starikovskaya}
\affil[1]{DI/ENS, PSL Research University, France}
\affil[2,3]{Institute of Computer Science, University of Wrocław, Poland}
\affil[4]{DI/ENS, PSL Research University, France}

\theoremstyle{plain}
\newtheorem{theorem}{Theorem}
\newtheorem{lemma}{Lemma}  
\newtheorem{fact}{Fact}
\newtheorem{corollary}[fact]{Corollary}  
\newtheorem{observation}{Observation}
\theoremstyle{definition}
\newtheorem{definition}{Definition}
\newtheorem{example}{Example}
\newtheorem{conjecture}{Conjecture} 
\newtheorem{claim}{Claim}
\newtheorem{problem}{Problem}
\theoremstyle{remark}
\newtheorem*{remark}{Remark}

%\crefname{claim}{claim}{claims}
\crefname{claim}{Claim}{Claims}

\def\dd{\mathinner{.\,.}}
\newcommand{\R}{\mathbb{R}}
\newcommand{\Z}{\mathbb{Z}}
\newcommand{\N}{\mathbb{N}}
\renewcommand{\O}{\mathcal{O}}
\newcommand{\tO}{\tilde{\mathcal{O}}}

\newcommand{\Q}{\mathcal{Q}}
\newcommand{\T}{\mathcal{T}}
\renewcommand{\S}{\mathcal{S}}
\renewcommand{\P}{\mathcal{P}}
\newcommand{\U}{\mathcal{U}}
\newcommand{\V}{\mathcal{V}}
\newcommand{\F}{\mathcal{F}}
\renewcommand{\L}{\mathcal{L}}

\renewcommand{\phi}{\varphi}
\newcommand{\floor}[1]{\left\lfloor #1 \right\rfloor}
\newcommand{\set}[1]{\left\lbrace #1 \right\rbrace}
\newcommand{\bigset}[1]{\big \lbrace #1 \big \rbrace}
\newcommand{\Bigset}[1]{\Big \lbrace #1 \Big \rbrace}
\newcommand{\eq}[1]{\begin{align*} #1 \end{align*}}

\DeclareMathOperator*{\Edges}{E}
\DeclareMathOperator*{\X}{X}
\DeclareMathOperator*{\Y}{Y}
\DeclareMathOperator*{\score}{score}
\DeclareMathOperator*{\Ham}{Ham}
\DeclareMathOperator*{\ID}{Id}
\DeclareMathOperator*{\dom}{dom}
\DeclareMathOperator*{\chrome}{C}

\newcommand{\absolute}[1]{\lvert#1\rvert}

\newcommand{\defproblem}[3]{
\vspace{2mm}
\noindent\fbox{
  \begin{minipage}{0.94\textwidth}
    \textsc{\large #1}\\
    {\bf{Input:}} #2  \\
    {\bf{Output:}} #3
  \end{minipage}
  }
\vspace{2mm}
}

\newcommand{\jonas}[2][]{\todo[color=green!40, #1]{\textbf{J:} #2}}
\newcommand{\jonasi}[2][]{\jonas[inline, #1]{#2}}

\sloppy

% BEGIN DOCUMENT
\begin{document}

\date{}
\maketitle

\begin{abstract}
	We consider a natural generalization of the classical approximate pattern matching problem to two-dimensional strings.
	A two-dimensional string is simply a square array of characters.
	Given two such arrays, the pattern of size $m\times m$ and the text of size $n\times n$, our goal is to find all locations in the text where the pattern matches with at most $k$ mismatches.
	This problem has been extensively studied for regular one-dimensional strings, and by now, we have a good understanding of the best possible time complexity as a function of $n$, $m$, and $k$.
	In particular, we know that for $k=\O(\sqrt{m})$, we can achieve quasi-linear time complexity [Gawrychowski and Uznański, ICALP 2018].
	Surprisingly, no similar statement is known for two-dimensional strings, as the asymptotically fastest algorithm works in $\O(kn^{2})$ time [Amir and Landau, TCS 1991].
	We improve on these bounds from 30 years ago with a non-trivial adaptation of tools used to tackle the one-dimensional version and design an $\tO((m^{2}+mk^{5/4})n^{2}/m^{2})$ time algorithm.
	In other words, our algorithm works in $\tO(n^{2})$ time for $k=\O(m^{4/5})$.
	The results described in this thesis have been obtained in a collaboration between Jonas Ellert, Paweł Gawrychowski, Adam Górkiewicz, and Tatiana Starikovskaya, and will form the basis of a later joint publication.
\end{abstract}

\section{Introduction}
The fundamental algorithmic problem considered in the context of sequences of characters, called strings, is pattern matching:
finding one string in another. Efficient linear-time algorithms for this problem are known since the 70s~\cite{Knuth1977}.
However, from the point of view of possible applications, it is desirable to search for approximate occurrences.
A clean and yet possibly useful in practice notion of an approximate occurrence is that of bounded Hamming distance,
where given a parameter $k$, we want to find all positions in the text where the pattern matches with at most $k$
mismatches. The natural assumption is that $k$ is not too large, and the running time should be close to linear
when $k$ is small. The first algorithms~\cite{Landau1986,Galil1986} that achieved such a goal in the 80s used the technique
informally called ``kangaroo jumping'': they consider each position in the text and calculate the number of mismatches
by jumping over regions where there is no mismatch. A single jump can be implemented in constant time with a
data structure for the longest common extensions, such as a suffix tree augmented with a lowest common ancestors structure,
and after having found more than $k$ mismatches we can move to the next position in the text. Thus, the overall
time becomes $\O(nk)$. For very large values of $k$ this is not better than the naive algorithm. However, another
approach based on the fast Fourier transform works in $\O(n \sqrt{m \log m})$ time~\cite{Abrahamson1987},
suggesting that the $\O(nk)$ bound is not optimal for the whole range of values of $k$. It was only in 2004 that
both bounds were unified to obtain an $\O(n\sqrt{k \log k})$ time algorithm~\cite{Amir2004}. This complexity
was later improved to $\tO(n + k^2n/m)$~\cite{Clifford2016a}, and then further refined to $\tO(n + kn/\sqrt{m})$~\cite{Gawrychowski2018},
which gives a smooth trade-off between $\tO(n\sqrt{k})$ and $\tO(n + k^2n/m)$\footnote{We write $\tO$ to hide factors polylogarithmic in $n$.}. 
It is known that a significantly faster algorithm implies fast boolean matrix multiplication~\cite{Gawrychowski2018},
and the time complexity can be slightly improved to $\O(n + kn\sqrt{(\log m) / m})$ \cite{Chan2020} (at the expense
of allowing Monte Carlo randomization). In a very recent exciting improvement, it was shown how to slightly improve
these time complexities by leveraging a connection to the 3-SUM problem~\cite{Chan0WX23}. Thus, the time complexity of one-dimensional pattern matching with bounded Hamming distance is fairly well understood.
This is also the case from the more combinatorial point of view: we know that occurrences of the pattern with $k$ mismatches
either have a simple and exploitable structure, or the pattern is close to being periodic~\cite{Bringmann2019,Charalampopoulos2020a}.

\paragraph{2D strings.} The natural extension of strings to two dimensions is to consider arrays
of characters, called 2D strings. To avoid multiplying the parameters, we will assume that they are square. Such an extension
is motivated by the possible application in image processing. Then, the basic algorithmic problem becomes
to find all occurrences of an $m\times m$ pattern in an $n\times n$ text. An efficient $\O(n^{2}+m^{2})$ time algorithm
for this problem was obtained already in the late 70s~\cite{Bird1977}, but obtaining such complexity without
any assumption on the size of the alphabet was achieved only in the mid-90s~\cite{Amir1994,Galil1996}
(even in logarithmic space~\cite{Crochemore1995}). Efficient parallel algorithms have also been obtained~\cite{Crochemore1998,Crochemore1998},
and the time complexity for random inputs, i.e., average time complexity, has been considered~\cite{Baeza-Yates1993,Tarhio1996,Kaerkkaeinen1999}.

\paragraph{Periodicities in 2D strings.} The fundamental combinatorial tool used for 1D strings is periodicity,
defined as follows. We say that $p$ is a period of $s[1 \dd n]$ when $s[i]=s[i+p]$, for all $i$ such that the expression is defined.
The set of all periods of a given string has a very simple structure~\cite{Fine1965}. For 2D strings, the notion of periodicity
becomes more involved~\cite{Amir1998}, but remains to be a powerful tool for exact pattern matching~\cite{Amir1992,Galil1996}.
Some purely combinatorial properties of two-dimensional periodicities have been studied~\cite{Mignosi2003,Gamard2017},
but generally speaking repetitions in two-dimensional strings are inherently more complicated than in one-dimensional strings.
For example, compressed pattern matching for two-dimensional strings becomes NP-complete~\cite{Berman2002}, see~\cite{Rytter2000}
for a more extensive discussion.
Another example, perhaps less extreme, are the bounds on two-dimensional runs~\cite{Amir2020} and
distinct squares~\cite{Charalampopoulos2020}, where we know that increasing the dimension incurs at least an additional
logarithmic factor~\cite{Charalampopoulos2020}.

\paragraph{2D pattern matching with $k$ mismatches.} The next step for 2D pattern matching is to allow $k$ mismatches.
Already in 1987, an $\tO(kmn^{2})$ time algorithm was obtained for this problem~\cite{Krithivasan1987}. This was
soon improved to $\tO((k+m)n^2)$ time~\cite{Ranka1991}, and finally to $\O(kn^2)$~\cite{Amir1991}, which remains
to be the asymptotically fastest algorithm. A number of non-trivial results have been obtained under the assumption
that the input is random, i.e. for the average time complexity~\cite{Baeza-Yates1998,Park1998,Kaerkkaeinen1999}.
Given that other notions of approximate occurrences, e.g. bounded edit distance, seem less natural in the two-dimensional
setting~\cite{Baeza-Yates1998a}, the natural challenge is to better understand the complexity
of 2D pattern matching with $k$ mismatches. In particular, it would be interesting to design a quasi-linear time
algorithm for polynomial $k=\O(n^{\epsilon})$ number of mismatches.

\paragraph{Our result.} We design an algorithm that, given an $n\times n$ text and $m\times m$ pattern, finds
all occurrences with at most $k$ mismatches of the former in latter in $\tO((m^2 + mk^{5/4})n^2 / m^2)$ time.
This significantly improves on the previously known upper bound of $\O(kn^{2})$ (from over 30 years ago),
and provides a quasi-linear time algorithm for $k=\O(m^{4/5})$. 

\paragraph{Overview of the techniques.}
The starting point for our algorithm is the approach designed for the one-dimensional version, see e.g.~\cite{Gawrychowski2018}
for an optimized version (but the approach is due to~\cite{Clifford2015}), which proceeds as follows.
First, we approximate the Hamming distance for every position in the text with Karloff's algorithm~\cite{Karloff1993}.
Then, we can eliminate positions for which the approximated distance is very large.
If the number of remaining positions is small enough, we can use kangaroo jumps~\cite{Galil1986} to verify them one by one.
Otherwise, some two remaining possible occurrences must have a large overlap, and thus induce a small approximate
the period in the pattern, i.e., an integer $p$ such that aligning the pattern with itself at a distance $p$ incurs few mismatches.
Then, (for $n=2m$), we can restrict our attention to the middle part of the text with the same approximate period $p$.
Then, both the pattern and the text compress very well under the simple RLE compression, if we rearrange their characters
by considering the positions modulo $p$. In other words, they can be both decomposed into few subsequences
of the form $i, i+p, i+2p, \ldots, i+\alpha p$ consisting of the same character.
By appropriately plugging in an efficient algorithm for approximate pattern
matching for RLE-compressed inputs, this allows us to obtain the desired time complexity.

In the two-dimensional case, there is no difficulty in adapting Karloff's algorithm or kangaroo jumps, which allows
us to focus on the case where there are two possible occurrences with a large overlap. Here, the two-dimensional
case significantly departs from the one-dimensional case in terms of technical complications. In 2D, a period is
no longer an integer but a pair of integers, i.e., a vector. However, to obtain a compressed representation of
a 2D string with small approximate period we actually need two such periods (with some additional properties) and not
just one. We show that two vectors with the required properties exist with some geometric considerations and
applying Dilworth's theorem. Then, we show that, similarly to the 1D case, they allow us to decompose the pattern
into nicely structured monochromatic pieces. There are $\O(k)$ such pieces, and each of them consists of positions
defined by some lattice of points restricted to a polygon. We call such a set of positions a subtile.
The next step is to similarly decompose the text. In 2D it is less clear what would be its middle part that
admits the same approximate period, however we can build on this idea to partition the relevant part of the text
into monochromatic pieces. Then, we consider each piece of the pattern and each piece of the text,
and convolve them to calculate their contribution to the number of mismatches. This can be done in $\tO(1)$
per pair of pieces if one of them admits some additional condition that we call being simple.
Thus, we actually need to partition the relevant part of the text into simple subtiles. A direct approach
results in too many pieces, and so we proceed in a more indirect way by introducing the notion of
a peripheral set of positions of the text. These positions interact with not too many positions in the pattern,
and can be convolved differently. All remaining positions of the text are partitioned into not too many simple subtiles,
and then we convolve every simple subtile from the text with every subtile from the pattern.
Finally, we sum up the number of mismatches for each relevant position in the text.



We assume the standard word-RAM model of computation with words of size $\Omega(\log n)$.


\begin{theorem}[Informal version of \cref{th:main}]
Given a two-dimensional $m \times m$ string $P$ and a two-dimensional $n \times n$ string $T$, where $m,n \in \Z^+$ and $m \le n$. There is an algorithm that solves the $k$-mismatch problem in $\tO((m^2 + mk^{5/4})n^2 / m^2)$ time.
\end{theorem}

%%%%%%%%%%%%%%%%%%%%%%%%%%%%%%%%%%%%%%%%%%%%
%%%%%%%%%%%%%%%%%%%%%%%%%%%%%%%%%%%%%%%%%%%%
%%%%%%%%%%%%%%%%%%%%%%%%%%%%%%%%%%%%%%%%%%%%
\section{Preliminaries}
\label{sec:preliminaries}
\newcommand{\x}[1]{#1.x}
\newcommand{\y}[1]{#1.y}
\newcommand{\h}[1]{\phi \times #1}
\newcommand{\s}[1]{\psi \times #1}

\textbf{Geometric notations.} For $n \in \Z^+$, denote $[n] = \set{0, \dots, n - 1}$. For $u \in \R^2$, denote its coordinates as $\x{u}, \y{u}$, i.e. $u = (\x{u}, \y{u})$. Furthermore, for $u, v \in \R^2$, denote 
\begin{align*}
&u + v = (\x{u}+\x{v}, \y{u} + \y{v})  & u + v = (\x{u}-\x{v}, \y{u}-\y{v})\\
&u \cdot v = \x{u} \cdot \x{v} + \y{u} \cdot \y{v}  & u \times v = \x{u} \cdot \y{v} - \y{u} \cdot \x{v}
\end{align*}
Alternatively, $u \cdot v = |u||v| \cos \alpha$ and $u \times v = |u||v| \sin \alpha$, where $\alpha$ is the angle between $u$ and $v$.

\begin{definition}
	For a set $U \subseteq \R^2$ we denote
	\[ \X(U) = \set{\x{u} : u \in U}, \quad \Y(U) = \set{\y{u} : u \in U}.\]
\end{definition}

%We will use the terms point and vector interchangeably.

\newcommand{\wild}{\texttt{?}}
\noindent \textbf{String notations.} In this work, $\Sigma$ denotes an \emph{alphabet}, a finite set consisting of integers polynomially bounded in the size of the input strings. The elements of $\Sigma$ are called \emph{characters}. Additionally, we consider a special character denoted by $\wild$ that is not in $\Sigma$ and is called a \emph{wildcard}.

\newcommand{\getchar}[1]{\chrome(#1)}
\newcommand{\pto}{\mathrel{\ooalign{\hfil$\mapstochar\mkern5mu$\hfil\cr$\to$\cr}}}
\renewcommand{\d}[1]{\dom(#1)}
\newcommand{\f}[1]{#1^\mathbf{f}}

A \emph{one-dimensional string} $S$ is a function $[n] \rightarrow \Sigma \cup \{\wild\}$. We denote its $i$-th character, $i \in [n]$, by $S(i)$, and sometimes we will simply write $S = S(1) S(2) \ldots S(n)$. For two-dimensional strings, we use a slightly more general definition: Namely, we define a \emph{two-dimensional string} $S$ as a partial function $\Z^2 \pto \Sigma$. We store strings as lists of point-character pairs. 

For a string $S$, let $\d{S}$ be its domain. The \emph{size} of $S$ is the size of its domain. The \emph{width} of a non-empty two-dimensional string $S$ is defined as $\max \X(\d(S)) - \min \X(\d(S))$ and the \emph{height} as $\max \Y(\d(S)) - \min \Y(\d(S))$. Finally, we say that $S$ is \emph{partitioned} into strings $R_1, \dots, R_\ell$ if $\d{S} = \sqcup_{i = 1}^\ell \d{R_i}$ and $R_i(u) = S(u)$ for all $1 \le \i \le \ell$ and $u \in \d{R_i}$.

\newcommand{\hd}{\textsc{HD1D}\xspace}
\newcommand{\HD}{\textsc{HD2D}\xspace}
\newcommand{\MI}{\textsc{MI}\xspace}

\begin{definition}[Hamming distance]
	Two characters $a,b \in \Sigma$ \emph{match} if $a = b$. The wildcard matches itself and all characters in $\Sigma$. 
	For strings $S, R$ (either both one-dimensional or both two-dimensional), define
	\begin{align*}
		&\MI(S, R) = \set{u : u \in \d{S} \cap \d{R}, S(u) \neq R(u)}|\\
		&\Ham(S,R) = |\MI(S,R)|
	\end{align*}			 
	We call the elements of $\MI(S,R)$ the \emph{mismatches} between $S$ and $R$.
\end{definition}

\begin{definition}[Shifting]
	For a set $V \subseteq \Z$ (resp., $V \subseteq \Z^2$) and $u \in \Z$ (resp., $u \in \Z^2$), denote $V + u := \set{v + u : v \in V}$.
	For a one-dimensional (resp., two-dimensional) string $S$ and $u \in \Z$ (resp., $u \in \Z^2$), define $S + u$ to be a string $R$ such that
	$\d{R} = \d{S} + u$ and $R(v) = S(v - u)$ for $v \in \d{R}$.
	Intuitively, we shift the domain of the string while maintaining the character values corresponding to its elements.
\end{definition}


\defproblem{The text-to-pattern Hamming distances problem}{One-dimensional (resp., two-dimensional) strings $P, T$}{$\Ham(T, P+q)$ for all $q \in \Z$ (resp., $q \in \Z^2$) such that $\d{P + q} \subseteq \d{T}$}

\defproblem{The $k$-mismatch problem}{One-dimensional (resp., two-dimensional) strings $P, T$, an integer $k \in \Z$}{$\min\{k+1,\Ham(T, P+q)\}$ for all $q \in \Z$ (resp., $q \in \Z^2$) such that $\d{P + q} \subseteq \d{T}$}

Traditionally, $P$ is referred to as \emph{the pattern} and $T$ as \emph{the text}.

\subsection{Generalisations of one-dimensional algorithms}
We start by generalising existing one-dimensional algorithms to two dimensions that we will use as subroutines in the final algorithm for the $k$-mismatch problem. 

For a two-dimensional string $S$ of width $w_S$ and height $h_S$ define its \emph{linearisation} as a one-dimensional string $\bar{S}$ of size $w_S \times h_S$, where $\forall x \in \X(\d{S}), y \in \Y(\d{S})$:

$$
\bar{S}((x-1) \cdot h_S + y) =
\begin{cases}
\wild & \text{ if } S(x,y) \text{ is not defined}\\
S(x,y) & \text{ otherwise}
\end{cases}
$$

Intuitively, this can be seen as inscribing $S$ into a rectangle, filling non-defined character values with wilcards, and writing the rectangle two-dimensional string row-by-row. 

\begin{theorem}\label{kangaroos}
Assume that $P, T$ are two-dimensional strings where $\d{P} = [m] \times [m]$ and $\d{T} = [n] \times [n]$ for $m \le n \in \Z^+$. Given a set $Q \subseteq \Z^2$, there is an algorithm that computes $d_q = \Ham(P + q, T) $ for every $q \in Q$ in total time $\O(n^2 + \sum_{q \in Q} d_q)$.
\end{theorem}
\begin{proof}
For every $i \in [m]$, define a one-dimensional string $P_i = P(i,1) P(i,2) \ldots P(i,m)$. Analogously, for every $i \in [n], j \in [n-m+1]$, define a one-dimensional string $T_{i,j} = T(i,j) T(i,j+1) \ldots T(i,j+m-1)$. 

We assign in $\O(n^2)$ time an integer identifier $\ID$ to each $P_i$ and $T_{i,j}$, satisfying the property that the identifiers are equal iff the strings are equal, in a folklore way: First, we build the suffix tree~\cite{DBLP:conf/focs/Weiner73} for $\bar{T} \$ \bar{P}$ in $\O(n^2)$ time, where $ \notin \Sigma \cup \{\wild\}$. Next, we trim the suffix tree at depth $m$ by a depth-first traverse in $\O(n^2)$ time as well. By definition, every string $P_i$ and $T_{i,j}$ now corresponds to a leaf of the tree, and distinct strings correspond to distinct leaves. We can now identify a string with the corresponding leaf. 

\begin{fact}[{\cite{Galil1986}}]\label{fact:kangaroo1D}
Given a one-dimensional string $S$ of size $\ell$ such that none of its characters is a wildcard. There is a data structure that can be built in $\O(\ell)$ time and for all pairs of strings $S_1 = S(i) S(i+1) \ldots S(i+\ell')$, $S_2 = S(j) S(j+1) \ldots S(j+\ell')$, where $1 \le i,j \le \ell-\ell'$, allows computing $\MI(S_1, S_2)$ in time $\O(\Ham(S_1,S_2))$. 
\end{fact}

Let $\ID(T) = \ID(T_{1,1}) \ID(T_{1,2}) \ldots \ID(T_{1,n-m+1}) \ldots \ID(T_{n,1}) \ID(T_{n,2}) \ldots \ID(T_{n,n-m+1}))$ and $\ID(P) = \ID(P_1)\ID(P_2) \ldots \ID(P_m)$. 
We build the data structure of \cref{fact:kangaroo1D} for two strings: $\ID(T) \$ \ID(P)$ and $\bar{T} \$ \bar{P}$ (note that neither of the strings contains a wildcard). In total, the construction takes $\O(|\ID(T) \$ \ID(P)| + |\bar{T} \$ \bar{P}|) = \O(n^2)$ time.

Consider now $q \in Q$. To compute $\Ham(T, P+q)$, note that 

\begin{align*}
&\Ham(T, P+q) = \sum_{i=1}^m \Ham(T_{\x{q}+i, \y{q}+1}, P_i)=\\
&\sum_{1 \le i \le m \; : \; \ID(T_{\x{q}+i, \y{q}+1}) \neq \ID(P_i)} \Ham(T_{\x{q}+i, \y{q}+1}, P_i) = \\
&\sum_{x \in \MI(\ID(T_{\x{q}+1, \y{q}+1}) \ldots \ID(T_{\x{q}+1, \y{q}+m}), \ID(P))} \Ham(T_{x, \y{q}+1}, P_{x-\x{q}})\\
\end{align*}

It follows that we can first use the data structure of \cref{fact:kangaroo1D} for  $\ID(T) \$ \ID(P)$ to compute the set $\MI =  \MI(\ID(T_{\x{q}+1, \y{q}+1}) \ID(T_{\x{q}+1, \y{q}+2}) \ldots \ID(T_{\x{q}+1, \y{q}+m}), \ID(P))$ in $\O(|\MI|) = \O(d_q)$ time, and then for each $x \in \MI$, we use the data structure of  \cref{fact:kangaroo1D}  for $\bar{T} \$ \bar{P}$ to compute 
$\Ham(T_{x, \y{q}+1}, P_{x-\x{q}})$ in total $\O(d_q)$ time as well. Summing up the resulting values, we obtain $\Ham(T, P+q)$. The claim follows. 
\end{proof}

\begin{fact}[Folklore corollary of \cite{CLIFFORD200753}]\label{fact:sigman1d}
Given one-dimensional strings $P$ of size $m$ and $T$ of size $n \ge m$ (both can contain wildcards), the text-to-pattern Hamming distances problem can be solved in $\O(\sigma \cdot n \log m)$ time, where $\sigma$ is the number of distinct characters present in $P$ and $T$.
\end{fact}

\begin{corollary}[of~\cite{Karloff1993}]\label{cor:approx1d}
Let $\varepsilon > 0$ be a constant. Given one-dimensional strings $P$ of size $m$ and $T$ of size $n \ge m$ (both can contain wildcards), there is a $(1+\varepsilon)$-approximation algorithm that solves the text-to-pattern Hamming distances problem in $\O((n/\varepsilon^2)  \log^3 m)$ time.
\end{corollary}
\begin{proof}
Karloff's algorithm~\cite{Karloff1993} solves the text-to-pattern Hamming distances problem for $P,T$ in $\O((n/\varepsilon^2)  \log^3 m)$ time assuming that neither $P$ nor $T$ does not contain a wildcard. We show how to modify it slightly to allow for mismatches. 

The idea of Karloff's algorithm is as follows. For each $1 \le i \le r = (\log m/ \varepsilon)^2$, the algorithm a mapping $\mu_i : \Sigma \rightarrow \{0,1\}$. It then solves the text-to-pattern Hamming distances problem for $\mu_i(T) := \mu_i(T(1))\mu_i(T(2)) \ldots \mu_i(T(n))$ and $\mu_i(P) := \mu_i(P(1))\mu_i(P(2)) \ldots \mu_i(P(m))$. For $1 \le j \le n-m$, define $d_i(j) = \Ham(\mu_i(T), \mu_i(P)+j-1)$. The algorithm returns the value $d(j) = \sum_i d_i(j)$. Karloff showed that $\Ham(T, P+j-1) \le d(j) \le (1+\varepsilon) \cdot \Ham(T, P+j-1)$.

We modify the mappings slightly. For $a \in \Sigma \cup \{\wild\}$, define
$$
\tilde{\mu}_i(a) = 
\begin{cases}
\mu_i(a) &\text{ if } a \in \Sigma\\
\wild & \text{ if } a = \wild.
\end{cases}
$$
We then construct $\tilde{\mu_i}(T)$ and $\tilde{\mu_i}(P)$ by applying the mapping to the characters of the strings similarly to above, and compute the values $\tilde{d}_i(j) = \Ham(\mu_i(T), \mu_i(P)+j-1)$ and $\tilde{d}(j) = \sum_i d_i(j)$ using \cref{fact:sigman1d} in $\O(r \cdot n \log m) = \O((n/\varepsilon^2) \log^3 m)$ time. It is not difficult to see that wildcards do not contribute to the distances neither between $T$ and $P$, nor between $\tilde{\mu_i}(T)$ and $\tilde{\mu_i}(P)$. Therefore, $\Ham(T, P+j-1) \le \tilde{d}(j) \le (1+\varepsilon) \cdot \Ham(T, P+j-1)$, and the algorithm is correct. 
\end{proof}

We now show a simple embedding from two-dimensional strings to one-dimensional, that is essential for generalising \cref{fact:sigman1d} and \cref{cor:approx1d} to two dimensions.

\begin{claim}
\label{claim:padding}
Given two non-empty two-dimensional strings $P$ and $T$ of widths $w_P, w_T$ and heights $h_P, h_T$. There are one-dimensional strings $\lambda(P)$ of size $w_T \times h_P$ and $\lambda(T)$ of size $w_T \times h_T$ that we can compute in $\O((w_P+w_T)(h_P+h_T))$ time such that to compute the text-to-pattern Hamming distances problem for $P$ and $T$ it is enough to solve it for $\lambda(P)$ and $\lambda(T)$. 
\end{claim}
\begin{proof}
We first construct a two-dimensional string $P'$ such that for all $\min \X(\d{P}) \le i \le \max \X(\d{P})+(w_T-w_P)$ and $\min \Y(\d{P}) \le j \le \max \Y(\d{P})$ there is
%
$$
P'(i,j) = 
\begin{cases}
P(i,j) & \text{ if } (i,j) \in \d{P}\\
\wild & { otherwise}
\end{cases}
$$
%
Intuitively, we inscribe $P$ into a rectangle, then add $w_T-w_P$ columns to it, and finally fill all non-defined character values with wildcards. Note that for all $q \in \Z^2$ there is $\Ham(T,P+q) = \Ham(T,P'+q)$. We linearise $T$ to obtain a one-dimensional string $\bar{T}$ of size $w_T \times h_T$ and $P'$ to obtain a one-dimensional string $\bar{P'}$ of length $w_T \times h_P$. We then have:
%
$$\Ham(T, P+q) = \Ham(T,P'+q) = \Ham(\bar{T}(\x{q} \cdot h_T + (\y{q}+1)) \ldots \bar{T}(\x{q} \cdot h_T + (\y{q}+w_T \times h_P)), \bar{P'})$$
%
The claim follows. 
\end{proof}

\begin{corollary}\label{cor:sigman2d}
Given two non-empty two-dimensional strings $P$ and $T$ of widths $w_P, w_T$ and heights $h_P, h_T$, we can solve the text-to-pattern Hamming distances problem in time $\O((\sigma+1) \cdot N \log N)$, where $\sigma$ is the number of different characters present both in $P$ and $T$ and $N = (w_P + w_T)(h_P + h_T)$.
\end{corollary}
\begin{proof}
We replace all characters of $T$ non-present in $P$ with a new character in $\O(w_T\times h_T)$ time. The Hamming distances do not change. The claim then follows immediately from \cref{fact:sigman1d} and \cref{claim:padding}. 
\end{proof}


\begin{corollary}\label{cor:approx2d}
Let $\varepsilon > 0$ be a constant. Given two non-empty two-dimensional strings $P$ and $T$ of widths $w_P, w_T$ and heights $h_P, h_T$, there is a $(1+\varepsilon)$-approximation algorithm that solves the text-to-pattern Hamming distances problem in time $\O((N/\varepsilon^2)  \log^3 N)$, where $N = (w_P + w_T)(h_P + h_T)$.
\end{corollary}
\begin{proof}
The claim then follows from \cref{cor:approx1d} and \cref{claim:padding}. 
\end{proof}

%Unfortunately, the most effective known algorithms for bounded \hd~\cite{Clifford2015,Gawrychowski2017} rely on periodicity and inherently do not allow don't care symbols, thus, they cannot be easily generalized.

%%%%%%%%%%%%%%%%%%%%%%%%%%%%%%%%%%%%%%%%%%%%
%%%%%%%%%%%%%%%%%%%%%%%%%%%%%%%%%%%%%%%%%%%%
%%%%%%%%%%%%%%%%%%%%%%%%%%%%%%%%%%%%%%%%%%%%
\section{Technical overview}
We can now give a formal statement of our result:

\begin{restatable}{theorem}{MainResult}\label{th:main}
Given two-dimensional strings $P$ with $\d{P} = [m] \times [m]$ and $T$ with $\d{T} = [n] \times [n]$, where $m,n\in Z^+$ and $m \le n$. There is an algorithm that solves the $k$-mismatch problem for $P,T$ in $\tO((m^2 + mk^{5/4})n^2 / m^2)$ time.
\end{restatable}

Below, we present the technical main ideas behind our result. Assume w.l.o.g. that $2|n$ and $n \le \frac{3}{2}m$. If this is not the case, one can cover $T$ with $\O(n^2/m^2)$ strings $T_i$ such that $\d{T_i} = [n'] \times [n'] + q$ for $n' = \Theta(m)$ and $q \in \Z^2$ and solve the problem for $P$ and each $T_i$ independently, which will result in $\O(n^2/m^2)$ extra multiplicative factor in the time complexity. Hence, below we assume that the two conditions are satisfied and aim to design an $\tO(m^2 + mk^{5/4})$-time algorithm. 

We start by applying \cref{cor:approx2d} to $P$, $T$, and $\varepsilon = 1$ and obtain in $\O(n^2 \log^3 n)$ time a set $Q \subseteq \Z^2$ such that both of the following hold:
\begin{enumerate}
\item For all $q \in \Z^2$ such that $\Ham(T, P+q) \le k$, we have $q \in Q$;
\item For all $q \in Q$, we have $\Ham(T, P+q) \le 2 k$.
\end{enumerate}

If $|Q| \le 6m + m^2/k$, we apply \cref{kangaroos} to compute the true value of the Hamming distance for all $q \in Q$, which takes $\O(mk+m^2)$ total time, completing the proof of \cref{th:main}. Below we assume $|Q| > 6m + m^2/k$. 

The idea of the algorithm is to partition the pattern $P$ and the text $T$ into strings of regular form each containing a single character. Intuitively, computing the Hamming distance between two such strings is easy: if the characters are equal, the distance is zero, and otherwise, its the area of their intersection.

% 2D PERIODICITY
\subsection{Two-dimensional periodicity} \label{periodicity_section}
We start by defining two-dimensional periods.

%\begin{figure}[!t]
%	\begin{center}
%		\includegraphics[width=0.8\textwidth]{drawings/occurrences}
%	\end{center}
%	\caption{Occurrences with small approximated Hamming distance. Red points belong to $Q$.}
%	\label{figure:occurrences}
%\end{figure}


\begin{definition}[Two-dimensional approximate period]
We say that a string $S$ has an \emph{$\ell$-period} $\delta \in \Z^2$ when $\Ham(S + \delta, S) \le \ell$.
\end{definition}


\begin{claim} \label{periodicity_lemma}
For all $u, v \in Q$, we have that $u - v \in \Z^2$ is a $\O(k)$-period of $P$.
\end{claim}	
\begin{proof}
$\Ham(P + u - v, P) = \Ham(P + u, P + v) \le \Ham(P + u, T) + \Ham(T,P + v) = \O(k)$.
\end{proof}

As $Q$ is big, it generates two $\O(k)$-periods with a big angle between them and such that they and their negations belong to distinct quadrants of $\Z^2$: 

\begin{restatable*}{theorem}{GetPeriods}\label{get_periods}
Given $\ell \in \Z^+$ and $U \subseteq [\ell + 1]^2$ of size $\ge 12\ell$. There is an $\tO(|U|)$-time algorithm that finds $u, v, u', v' \in U$, such that the following conditions hold for $\psi = u - v$ and $\phi = u' - v'$:
	\begin{itemize}
		\item $\psi \in (0, +\infty) \times [0, +\infty)$, $\phi \in [0, +\infty) \times (-\infty, 0)$;
		\item $0 < |\psi||\phi| = \O(\ell^2 / |U|)$,
		\item $\sin \alpha \ge \frac{1}{2}$, where $\alpha$ is the angle between $\psi$ and $\phi$.
\todo{update the proof}
	\end{itemize}
\end{restatable*}

Let $\ell = n - m \le m / 2$. We apply \Cref{get_periods} on $\ell$ and the set $Q$ (the conditions of the claim are satisfied as $|Q| > 6m + m^2/k \ge 12\ell$) and compute $\phi, \psi \in \Z^2$ in $\O(|Q|)$ time. By \Cref{periodicity_lemma}, $\phi$ and $\psi$ are $\O(k)$-periods of $P$, and $0 \le \phi \times \psi \le |\phi||\psi| = \O(\ell^2 / |Q|) =  \O(\min\set{m, k})$. We fix $\phi$ and $\psi$ for the rest of the paper.

\subsection{Partitionings of the pattern and the text}
Next, we define tiles, which will be used to partition $P$ and $T$ into one-character strings of regular form.

\begin{definition}[Lattice congruency]\label{lattice_congruency}
We define \emph{a lattice} $\L = \set{s\phi + t\psi : s, t \in \Z}$. We say that $u, v \in \Z^2$ are \emph{congruent}, denoted $u \equiv v$, if $u - v \in \L$. 
\end{definition}

\begin{figure}[h!]
	\begin{center}
		\includegraphics[width=0.8\textwidth]{drawings/parquet}
	\end{center}
	\caption{All the points in the polygon form a tile and the red points form a subtile.}
	\label{figure:tile}
\end{figure}

\begin{definition}[Tile]\label{tile_definition}
We call $U \subseteq \Z^2$ a \emph{tile} if there exist $\phi_0, \phi_1, \psi_0, \psi_1 \in \Z$, such that
	\[ U = \set{u : u \in \Z^2, \h{u} \in [\phi_0, \phi_1], \s{u} \in [\psi_0, \psi_1]}. \]
We call the values $\phi_0, \phi_1, \psi_0, \psi_1$ defining $U$ its \emph{signature}. 
\end{definition}

Geometrically, a tile $U$ can be viewed as a set of integer points inside a parallelogram with sides collinear to $\phi$ and $\psi$. Truncated tiles are a generalisation of simple tiles. 

\begin{definition}[Truncated tile]
We call $U \subseteq \Z^2$ a \emph{truncated tile} if there exist values $x_0, x_1, y_0, y_1 \in \Z \cup \{-\infty, +\infty\}$ and $\phi_0, \phi_1, \psi_0, \psi_1 \in \Z$ such that 
	\begin{enumerate}
		\item $U = [x_0, x_1] \times [y_0, y_1] \cap \set{u : u \in \Z^2, \h{u} \in [\phi_0, \phi_1], \s{u} \in [\psi_0, \psi_1]}$. 
		\item $x_1 - x_0 + 1 \ge |\x{\phi}| + |\x{\psi}|$ and $y_1 - y_0 + 1 \ge |\y{\phi}| + |\y{\psi}|$. 
	\end{enumerate}
The values $x_0, x_1, y_0, y_1, \phi_0, \phi_1, \psi_0, \psi_1$ defining $U$ are called its \emph{signature}. 
\end{definition}

Geometrically, the first property of a truncated tile $U$ means that it is a set of integer points in the intersection of an axis-parallel rectangle (possible, infinite) and a parallelogram with sides collinear to $\phi$ and $\psi$. The second property implies that the axis-parallel rectangle defined by $x_0, x_1, y_0, y_1$ contains a parallelogram spanned by $\phi$ and $\psi$. It will become clear later why this property is important. 

\begin{definition}[Subtile]\label{subtile_definition}
We call a set $V \subseteq \Z^2$ a \emph{(truncated) subtile} if there exists a (truncated) tile $U$ and $\gamma \in \Z^2$ such that $V = \set{u : u \in U, u \equiv \gamma}$. A signature of $V$ consists of the signature of $U$ and $\gamma$.
%We say that $V$ is lattice-congruent to some $v \in \Z^2$ (denoted as $V \equiv v$) when $v \equiv \gamma$. We similarly define the lattice congruency between two subtiles.
\end{definition}

See \Cref{figure:tile} for an illustration.

\begin{definition}[Tile string]\label{tile_string_definition}
We call a string $S$ a \emph{(truncated) (sub)tile string} if $\d{S}$ is a (truncated) (sub)tile.
\end{definition}

\begin{definition}[Monochromatic string]
A string $S$ is called \emph{monochromatic} if there is a character $\getchar{S} \in \Sigma$ such that for every $u \in \d{S}$ there is $S(u) = \getchar{S}$.  
\end{definition}

\begin{restatable*}{theorem}{tileDecomposition}\label{tile_decomposition}
Assume to be given a (truncated) tile string $R$ such that $\phi, \psi$ are its $\O(k)$-periods. After $\O(m^2)$-time preprocessing of $\phi, \psi$, the string $R$ can be partitioned in time $\tO(|\d{R}| + k)$ into $\O(k)$ monochromatic (truncated) subtile strings.
\end{restatable*}

Since $|\x{\phi}|, |\y{\phi}|, |\x{\psi}|, |\y{\psi}| \le n - m \le m / 2$, the string $P$ is a truncated tile string and we can apply \Cref{tile_decomposition} to partition it in time $\tO(m^2 + k)$ into a set of monochromatic truncated tile strings $\V$. We then group the strings in $\V$ based on the single character they contain. Specifically, for every character $\sigma \in \Sigma$ present in $P$, we construct a set $\V_\sigma = \set{V : V \in \V, \getchar{V} = \sigma}$.


\todo{T: I changed spacious parquet to tile, and simple parquet to tile. Parquet itself was never used. I need to update everything.}

Because $T$ is not necessarily periodic, we are obliged to use a similar but more sophisticated partitioning.

\newcommand{\Ta}{T_\mathbf{a}}
\begin{definition}[Active text]
We define the \emph{active text} $\Ta$ as the restriction of $T$ to $\bigcup_{q \in Q} \d{P + q}$. 
\end{definition}

\begin{observation}\label{obs:active_text}
$\Ham(P + q, T) = \Ham(P + q, \Ta)$ for every $q \in Q$.
\end{observation}

\begin{definition}[Peripherality]
For every point $u \in \Z^2$ we define its \emph{border distance} as $\min\set{|u - v| : v \in \Z^2 \setminus \d{\Ta}}$. A set $U \subseteq \Z^2$ is $d$-\emph{peripheral} for some $d \ge 0$, if the border distance of every $u \in U$ is at most $d$. We say that a string $S$ is $d$-peripheral when $\d{S}$ is $d$-peripheral.
\end{definition}

\begin{restatable*}{theorem}{TextDecomposition}\label{text_decomposition}
Given $\ell \in \Z^+$, one can partition $T_a$ in time $\tO(m^2 + \ell k)$ into $\O(\ell k)$ monochromatic subtile strings and a $\O(m / \ell)$-peripheral string.
\end{restatable*}

\subsection{Algorithm}
In \cref{sec:algorithm}, we show that the Hamming distances between $P$ and a string consisting of a monochromatic subtile strings can be computed efficiently: 

\begin{restatable*}{theorem}{SparseAlgo}
\label{th:sparse_algo}
Let $\S$ be a set of monochromatic subtile strings $\S$ with a property the domains $\d{S}$ for $S \in \S$ are pairwise disjoint subsets of $\d{T}$. There is an algorithm that computes
$\sum_{S \in \S} \Ham(P + q, S)$ for every $q \in Q$ in total time $\tO(m^2 + \sum_{S \in \S} |\V_{\getchar{S}}|)$.
\end{restatable*}

As an immediate corollary, we obtain an $\tO(m^2 + mk^2)$-time solution for the $k$-mismatch problem. First, we apply \Cref{text_decomposition} for a large enough value $\ell = \Theta(m)$ to partition the active text $\Ta$ into a set $\S$ of $\O(mk)$ monochromatic subtile strings and a $0$-peripheral string, empty by definition, in time $\tO(m^2 + mk)$. For every $q \in Q$ we then have $\Ham(P + q, \Ta) = \sum_{S \in \S} \Ham(P + q, S)$, and these values can be computed in $\tO(m^2 + mk^{2})$ total time by \cref{sparse_algo}, since $\sum_{S \in \S} |\V_{\getchar{S}}| \le |\S| |\V| = \O(mk \cdot k)$. 

To improve the complexity further, we partition the active text using the algorithm from \Cref{text_decomposition} with $\ell = mk^{-3/4}$.
We obtain a set~$\S$ of $\O(mk^{1/4})$ monochromatic subtile strings $\S$, and a $\O(k^{3 / 4})$-peripheral string $F$. For every $q \in Q$ we then have

\[ \Ham(P + q, \Ta) = \Ham(P + q, F) + \sum_{S \in \S} \Ham(P + q, S).\]

By \cref{th:sparse_algo}, we can compute $\sum_{S \in \S} \Ham(P + q, S)$ for every $q \in Q$ in time $\tO(m^2 + mk^{5/4})$, since similarly to above we have $\sum_{S \in \S} |\V_{\getchar{S}}| \le |\S| |\V| = \O(mk^{5/4})$. To compute the values $\Ham(P + q, F)$, we apply the following result:

\begin{restatable*}{theorem}{DenseAlgo}
\label{th:dense_algo}
Given a $d$-peripheral string $F$, there is an algorithm which computes $\Ham(P + q, F)$ for every $q \in Q$ in total time $\tO(m^2 + mdk^{1/2})$.
\end{restatable*}

By substituting $d = \O(k^{3/4})$, we obtain the desired time complexity of $\tO(m^2 + mk^{5/4})$ and hence complete the proof of \cref{th:main}.

%%%%%%%%%%%%%%%%%%%%%%%%%%%%%%%%%%%%%%%%%%%%
%%%%%%%%%%%%%%%%%%%%%%%%%%%%%%%%%%%%%%%%%%%%
%%%%%%%%%%%%%%%%%%%%%%%%%%%%%%%%%%%%%%%%%%%%
\section{Two-dimensional periodicity: Proof of \cref{get_periods}}
In this section, we show the following result that allows us to select two ``good'' periods of the pattern:

%\def\newrel{\le_{\rho, z}}
%\begin{lemma}
%Let $\ell \in \mathbb Z^+$ and $U \subseteq [\ell + 1]^2$. Let $z \in [0, \pi/2]$, $\rho \in \mathbb Z^2$, and let $R = [-z, +z]$. Let $\newrel$ be a relation defined as follows. For every $u \in U$, it holds $u \newrel u$. For $u, v \in U$ with $u \neq v$, it holds $u \newrel v$ if and only if the angle between $(v - u)$ and $\rho$ is within $R$. Then $\newrel$ is a partial order.
%\end{lemma}

\GetPeriods
\begin{proof}
We start by finding the closest pair of points in $U$.
Specifically, we select any pair of different points $s, t \in U$, which minimizes $|t - s|$. 
Such a pair can be obtained in $\O(|U| \log |U|)$ time~\cite{Clarkson1983}.
We construct $w = t - s$.
%
Denote the quadrants of the plane as follows:%
%
\begin{align*}
&\Q_1 = (0,\ell+1] \times [0,\ell+1], \quad & \Q_2 = [-(\ell+1),0] \times (0,\ell+1),\\
&\Q_3 = [-(\ell+1),0) \times [-(\ell+1),0], \quad & \Q_4 = [0,\ell+1] \times [-(\ell+1),0).
\end{align*}
%
We will compute a vector $w'$ with the following properties. If $w \in \Q_i$, then $w' \in \Q_{(i \bmod 4) + 1}$, i.e., $w'$ is in the counterclockwise neighbouring quadrant of $w$. Further, it holds $0 < |w||w'| = \O(\ell^2 / |U|)$, and $\sin \alpha \ge \frac{1}{2}$, where $\alpha$ is the angle between $w$ and $w'$. Depending on $i$, we then report $\psi$ and $\phi$ as follows. If $i = 1$, then $\psi = w$ and $\phi = -w'$. If $i = 2$, then $\psi = -w'$ and $\phi = -w$. If $i = 3$, then $\psi = -w$ and $\phi = w'$. If $i = 4$, then $\psi = w'$ and $\phi = w$. Clearly, this satisfies the claimed properties.


Assume w.l.o.g.\ that $w \in \Q_1$ (the other cases are symmetric). 
Let $\rho = ({\sqrt{2}} / 2,{\sqrt{2}} / 2)$ be the diagonal that splits $\Q_1$ in half, truncated to unit length.
We define a relation $\le_{w}$ on $\Z^2$ as follows. First, $u \le_w u$ for all $u \in \Z^2$ (i.e., the relation is \emph{reflexive}). Second, for $u, v \in \Z^2$ with $u \neq v$, we have $u \le_w v$ if and only if the angle $\beta$ between $\rho$ and $\delta = (v - u)$ is in $R = [-5\pi/12, 5\pi/12]$. Geometrically speaking, this is the case if $\delta$ points into $\Q_1$ extended with the respective adjacent 30 degrees of $\Q_2$ and $\Q_4$, which is visualized in \cref{TODO}.

Assume that there is a third vector $v' \in \Z^2$ such that $u \leq v \leq_{w} v'$ and $u \neq v \neq v'$. Let $\delta' = (v' - v)$, then the angle $\beta'$ between $\rho$ and $\delta'$ is in $R$.
Due to $\{\beta, \beta'\} \subseteq R$, and since $R$ is a consecutive range, it is clear that the angle $(\beta + \beta')/2$ between $(v' - u) = \delta + \delta'$ and $\rho$ is also in $R$. Thus, it holds $u \le_{w} v'$ and the relation $\le_{w}$ is \emph{transitive}.
It is also an \emph{antisymmetric} relation as it cannot be that both $\beta$ and $\pi-\beta$ belong to~$R$. We have shown that $\le_w$ is a partial order.
Note that, if $u \le_{w} v$ with $u \neq v$, then $\delta = (v - u)$ satisfies
%
$$%
%2\sqrt{2} \cdot (\delta.x + \delta.y) = 
4 \cdot (\delta \cdot \rho) \geq 
4 \cdot \cos(5\pi/12)\absolute{\delta}\absolute{\rho} = 
4 \cdot \cos(5\pi/12)\absolute{\delta} > \absolute{\delta} \geq \absolute{w},$$
%
where the last inequality holds by design of $w$. In a moment, we will show that we can compute the longest chain $C$ and antichain $A$ with respect to $U$ and $\le_{w}$ in $\tO(\absolute{U})$ time. Now we show that, given $A$ and $C$, we can find the vector $w'$ with the desired properties in $\tO(U)$ time.

%Hence, all $\delta \in \Z^2$ satisfying (\ref{joint_condition}) belong to a single half-plane and $\delta \cdot \rho > \cos(5\pi / 12)|\delta||\rho| > |\delta| / 4$.

%Under the partial order $\le_w$, we find the longest chain $C$ and the longest antichain $A$ using dynamic programming.
%With an appropriate data structure we can do it in $\tO(|U|)$ operations, by using a method resembling LIS calculation. \todo{I couldn't find a reference. Haven't touched what's below.}

\begin{claim}\label{C_ineq}
	$\absolute{C} \cdot \absolute{w} < 12\ell$. 
	\begin{proof}
		Let $f = |C| - 1$ and
		let $c_0, \dots, c_{f}$ denote the points in $C$, ordered so that $c_i \le_w c_{i + 1}$ for every $i \in [f]$.
		Consider the sequence $\delta_0, \dots, \delta_{f - 1}$, where $\delta_i = c_{i + 1} - c_i$ for every $i \in [f]$.
		We have already shown $4 \cdot (\rho \cdot \delta_i) > \absolute{w}$. Since dot-product commutes with addition, it holds
		$$f \cdot \absolute{w} < 4 \cdot (\rho \cdot \sum_{i = 0}^{f - 1} \delta_i) = 4 \cdot (\rho \cdot (c_f - c_0)) = 2\sqrt{2} \cdot (c_{f}.x - c_{0}.x + c_{f}.y - c_{0}.y) \leq 2\sqrt{2} \cdot 2\ell < 6\ell.$$%
		%
		We know that $|C| \ge 2$ because there is a chain containing $s$ and $t$. Hence $\absolute{C} - 1 = f \geq \absolute{C}/2$, and $f \cdot \absolute{w} < 6\ell$ implies $\absolute{C}\cdot\absolute{w} < 12\ell$.
	\end{proof}%
\end{claim}

\begin{lemma}[Dilworth's theorem]\label{dilworth}
	$|U| \le |C| |A|$.
\end{lemma}

By the assumption $|U| \ge 12\ell$ and \Cref{dilworth}, it holds
\[12 \ell < |U| \le |C| |A| < 12\ell|A|,\]
thus $|A| > 1$, which means $|A| \ge 2$.
We select any pair of different vectors $s', t' \in A$ that minimizes $|t' - s'|$. If $s'.x \leq t'.x$, then we construct $w' = s' - t'$. Otherwise, we construct $w' = t' - s'$.
We will now show that $|w||w'| = \O(\ell^2 / |U|)$.

\begin{claim}\label{A_ineq}
	$\absolute{A}\cdot \absolute{w'} < 12\ell$. 
	\begin{proof}
		We define the range $R' = (5\pi/12, 7\pi/12)$, and a corresponding relation $\le_{w}'$ defined such that $u \le_{w}' u$ for any $u \in U$, and $u \le_{w}' v$ for $u, v \in U$ with $u \neq v$ if and only if the angle between $\rho$ and $(v - u)$ is in $R'$. By the same reasoning as above (for $\le_w$), we can argue that $\le_{w}'$ is a partial order.
		Consider any $u, v \in U$ with $u \neq v$, and let $\beta$ be the angle between $(v - u)$ and $\rho$.
		If $\beta \in R \cup R' = [5\pi/12, 7\pi/12) = [0-5\pi/12, \pi - 5\pi/12)$, i.e., $(v - u)$ points into the half-plane defined by the range of angles, then it clearly holds exactly one of $u \le_w v$ and $u \le_w' v$, and none of $v \le_w u$ and $v \le_w' u$. Otherwise, $(u - v)$ points into the half-plane defined by the ranges, and it holds exactly one of $v \le_w u$ and $v \le_w' u$, and none of $u \le_w v$ and $u \le_w' v$. This means that any antichain with respect to $\le_{w}$ is a chain with respect to $\le_{w}'$ and vice versa.
		
		If $u, v \in U$ with $u \neq v$ satisfy $u \le_w' v$, then $\delta = (v - u)$ satisfies
		%
		$$%
4 \cdot (\delta \cdot \rho) \geq 
4 \cdot \cos(5\pi/12)\absolute{\delta}\absolute{\rho} = 
4 \cdot \cos(5\pi/12)\absolute{\delta} > \absolute{\delta} \geq \absolute{w}',$$	
		%
		where the last inequality holds by design of $w'$. Hence we can simply reproduce the proof of \cref{C_ineq} for $\le_w'$ and show that the longest chain $C'$ with respect to $\le_w'$ satisfies $\absolute{C'}\cdot \absolute{w'} < 12\ell$. This concludes the proof because $C'$ is the longest antichain with respect to $\le_w$.
	\end{proof}
\end{claim}

By multiplying the inequalities from \cref{A_ineq,C_ineq}, we obtain $\absolute{A}\absolute{C}\absolute{w}\absolute{w'} < 144\ell^2$. Finally, applying \cref{dilworth} and dividing both sides by $\absolute{U}$, we obtain the final statement $\absolute{w}\absolute{w'} < 144\ell/\absolute{U}$.

It remains to be shown that $w' \in \Q_2$, and how to compute $A$ and $C$ in $\tO(\absolute{U})$ time. \jonasi{work in progress.. }
\end{proof}

%%%%%%%%%%%%%%%%%%%%%%%%%%%%%%%%%%%%%%%%%%%%
%%%%%%%%%%%%%%%%%%%%%%%%%%%%%%%%%%%%%%%%%%%%
%%%%%%%%%%%%%%%%%%%%%%%%%%%%%%%%%%%%%%%%%%%%
\section{Partitioning of the pattern and the text}
The goal of this section is to show \cref{tile_decomposition} (which explains how to partition the pattern into monochromatic truncated subtile strings) and \cref{text_decomposition} (which explains how to partition the text). We start with a technical \cref{sec:lattices_subtiles}, which describes necessary properties of lattices and subtiles. 
 
% tile DECOMPOSITION
\subsection{Properties of lattices and subtiles} \label{sec:lattices_subtiles}
In this section we explore the properties of periodic (sub-)tile strings (recall Definitions \ref{tile_definition}, \ref{subtile_definition}, \ref{tile_string_definition}).
Specifically, we introduce some methods of partitioning them into monochromatic strings, which we utilize when decomposing both the pattern and the active text.

\begin{lemma} \label{lattice_base}
	There exists an algorithm that constructs in $\O(m^2)$ time a set $\Gamma \subseteq \Z^2$ such that $|\Gamma| = \O(\min\set{m, k})$ and $\forall u \in \Z^2 \; \exists \gamma \in \Gamma : u \equiv \gamma$.
\end{lemma} 
	\begin{proof}
		Let $p = \set{s\phi + t\psi : s \in [0, 1), t \in [0, 1)}$ and $\Gamma = p \cap \Z^2$.
		By Pick's Theorem, a simple polygon with integer vertices contains $\O(A)$ integer points in the interior or on the boundary, where $A$ denotes its area.
		Observe that the elements of $\Gamma$ are contained in a parallelogram with vertices $(0, 0), \phi, \phi + \psi, \psi$.
		Since its area is $\phi \times \psi = \O(\min\set{m, k})$, we get $|\Gamma| = \O(\min\set{m, k})$.
		
		Now consider any $u \in \Z^2$.
		As $\phi, \psi$ are not collinear, there exist unique values $s, t \in [0, 1)$ and $s', t' \in \Z$, such that
		$u = (s + s') \phi + (t + t') \psi$.
		By definition, 
		$u \equiv s\phi + t\psi$ and $s\phi + t\psi \in \Gamma$.
		
		We now give a construction algorithm for $\Gamma$. Consider $\Gamma' = \Z^2 \cap [0,\x{\phi}+\x{\psi}] \times [\y{\psi},\y{\phi}]$. We have $|\Gamma'| = \O((|\x{\phi}|+|\x{\psi}|) \cdot (|\y{\phi}|+|\y{\psi}|)) = \O(m^2)$ as $|\x{\phi}|, |\y{\phi}|, |\x{\psi}|, |\y{\psi}| \le n - m \le m / 2$. For $u \in \Gamma'$ we have $u \in \Gamma$ iff $0 \le \s{u} \le \s{\phi}$ and $0 \le \h{u} \le \h{\psi}$, which can be checked in constant time. 
	\end{proof}

\begin{definition}[Lattice graph]
	For a set $U \subseteq \Z^2$ we define its \emph{lattice graph} $(U, \Edges(U))$, where
	\[ \Edges(U) = \bigset{\set{u, u + \delta} : \delta \in \set{\phi, \psi}, u \in U, u + \delta \in U}\].
\end{definition}
Intuitively, we connect every $u \in U \cap Z^2$ with its translations by $\phi, \psi, -\phi, -\psi$ if they are contained in $U$.

\begin{lemma}\label{lattice_graph_connectivity}
	If $U$ is a truncated subtile, then $(U, \Edges(U))$ is connected.
	\begin{proof}
		Assume the contrary.
		Consider $u, v = u + s \phi + t \psi \in U$ such that
		\begin{itemize}
			\item $u$ and $v$ belong to different connected components of $(U, \Edges(U))$,
			\item $|s| + |t|$ is minimized.
		\end{itemize}
		W.l.o.g $s \ge 0$, since in the other case $u$ and $v$ can be swapped.
		We now show that there exists $w \in U$, such that $\set{u, w} \in \Edges(U)$ and if we let $s', t' \in \Z$ be such that $v = w + s'\phi + t'\psi$, then $|s'| + |t'| < |s| + |t|$, which contradicts the minimality of $|s| + |t|$. 
	
		Let $x_0, x_1, y_0, y_1, \phi_0, \phi_1, \psi_0, \psi_1 \in \Z$ and $\gamma \in \Z^2$ be the signature of $U$, meaning that
		\begin{itemize}
			\item $ U = [x_0, x_1] \times [y_0, y_1] \cap \set{z : z \in \Z^2, \h{z} \in [\phi_0, \phi_1], \s{z} \in [\psi_0, \psi_1], \gamma \equiv u}$,
			\item $x_1 - x_0 + 1 \ge |\x{\phi}| + |\x{\psi}|$ and $y_1 - y_0 + 1 \ge |\y{\phi}| + |\y{\psi}|$.
		\end{itemize}
		
		Recall that $\x{\phi} \ge 0$, $\y{\phi} \le 0$, $\x{\psi} \ge 0$, $\y{\psi} \ge 0$. We have the following cases:
		\begin{enumerate}[(a)]
			\item \label{it:s0_tneg} If $s = 0$, then w.l.o.g. (up to swapping $u$ and $v$) $t > 0$. Let $w := u + \psi$. Observe that
				\eq{
					\x{u} \le \x{w} \le \x{v}, \quad \y{u} \le \y{w} \le \y{v}, \quad \h{u} \le \h{w} \le \h{v}, \quad \s{w} = \s{u},
				}
				and since $w \equiv u$, we get $w \in U$. 
			
			\item If $s > 0$ and $t = 0$, let $w := u + \phi$. Analogously, 
				\eq{
					\x{u} \le \x{w} \le \x{v}, \quad \y{u} \le \y{w} \le \y{v}, \quad \h{u} \le \h{w} \le \h{v}, \quad \s{w} = \s{u},
				}
				and since $w \equiv u$, we get $w \in U$. 
			
			\item If $s > 0$ and $t > 0$, let $w' := u + \phi$. \label{case_pos}
				We have \eq{
					\x{u} \le \x{w'} \le \x{v}, \quad \h{w'} = \h{u}, \quad \s{v} \le \s{w'} \le \s{u}.
				}
				If $w' \in U$, let $w = w'$.
				If $w' \not \in U$, then since all other requirements are satisfied, we must have $\y{w'} \not \in [y_0, y_1]$.
				Since $\y{w'} = \y{u} + \y{\phi} \le \y{u}$, we have $\y{u} + \y{\phi} \le y_0 - 1$, and
				considering $y_1 - y_0 + 1 \ge |\y{\phi}| + |\y{\psi}|$, we get $y_1 \ge \y{u} + \y{\psi}$.
				Now let $w =: u + \psi$.
				We have \eq{
					\x{u} \le \x{w} \le \x{v}, \quad \y{u} \le \y{w} = \y{u} + \y{\psi} \le y_1, \quad \h{u} \le \h{w} \le \h{v}, \quad \s{w} = \s{u},
				}
				thus $w \in U$.
			\item If $s > 0$ and $t < 0$, consider $w' = u + \phi$.
				We have \eq{
					\y{v} \le \y{w'} \le \y{u}, \quad \h{w'} = \h{u}, \quad \s{v} \le \s{w'} \le \s{u}.
				}
				If $w' \in U$, then $w = w'$.
				Otherwise we can (similarly to (\ref{case_pos})) show that $w = u - \psi \in U$, by the fact that $x_1 - x_0 + 1 \ge |\x{\phi}| + |\x{\psi}|$. 
		\end{enumerate}
		In all these cases, which are clearly exhaustive, either $|s| \ge |s'| = 0$ and $|t| > |t'| = 1$ or $|s| > |s'| = 1$ and $|t| \ge |t'| = 0$ and hence $|s'|+|t'| \le |s|+|t|$, a contradiction. 
	\end{proof}
\end{lemma}


\begin{figure}[!t]
	\begin{center}
		\includegraphics[width=0.8\textwidth]{drawings/parquet_decomposition}
	\end{center}
	\caption{The partitioning of a subtile string into monochromatic subtile strings. Different colors represent different characters.}
	\label{figure:tile_decomposition}
\end{figure}


\begin{lemma}\label{monochromacy_condition}
	A truncated subtile string $S$ is monochromatic iff
	\[\Ham(S + \phi, S) + \Ham(S + \psi, S) = 0.\]
	\begin{proof}
		If $S$ is monochromatic, then clearly $\Ham(S + \phi, S) + \Ham(S + \psi, S) = 0$.
		Assume the contrary for the other implication.
		Let $u, v \in \d{S}$ be such that $S(u) \neq S(v)$.
		Since $\d{S}$ is a truncated subtile, the graph $(\d{S}, \Edges(\d{S})$ is connected by \Cref{lattice_graph_connectivity} and there must exist a path between $u$ and $v$.
		On that path there must exist a pair of neighbors $w, w'$, such that $S(w) \neq S(w')$ and $w' = w + \delta$ for some $\delta \in \set{\phi, \psi}$.
		If $\delta = \phi$, then $\Ham(S + \phi, S) \ge 1$ and if $\delta = \psi$, then $\Ham(S + \psi, S) \ge 1$ and we get a contradiction.
	\end{proof}
\end{lemma}


\begin{lemma}\label{cut_partitioning}
Given a truncated subtile string $S$, there is an algorithm that runs in time $\tO(|\d{S}| + 1)$ and constructs the following two partitionings.
	\begin{enumerate}[(a)]
		\item A partitioning of $S$ into a set of $\O(\Ham(S + \phi, S) + 1)$ (truncated) subtile strings $\U$, such that $\Ham(U + \phi, U) = 0$ for each $U \in \U$ and \label{partition_a}
		\item A partitioning of $S$ into a set of $\O(\Ham(S + \psi, S) + 1)$ strings $\V$, such that $\Ham(V + \psi, V) = 0$ for each $V \in \V$. \label{partition_b}
	\end{enumerate}
	\begin{proof}
		First, consider option (\ref{partition_a}). We construct the set
		\[ A = \set{\s{u} : u \in \d{S}, u + \phi \in \d{S}, S(u) \neq S(u + \phi)} \cup \set{-\infty, +\infty}\]
		and then sort its elements increasingly, creating a sequence $a_0, \dots, a_\ell$.
		Note that $\ell \le \Ham(S + \phi, S) + 2$.
		We then construct the strings $S_0, \dots, S_{\ell - 1}$, where $S_i$ is the restriction of $S$ to
		$\set{u : u \in \d{S}, \s{u} \in [a_i, a_{i + 1})}$ for every $i \in [\ell]$.
		Observe that $S_0, \dots, S_{\ell - 1}$ partition $S$ and that $\Ham(S_i + \phi, S_i) = 0$ for every $i \in [\ell]$. Finally, if $S$ is a (truncated) subtile string, then the created strings are (truncated) subtile strings as well.
	
		Second, consider option (\ref{partition_b}). Similarly to above, we construct
		\[ A = \set{\h{u} : u \in \d{S}, u + \psi \in \d{S}, S(u) \neq S(u + \psi)} \cup \set{-\infty, +\infty} \]
		and then sort it increasingly, creating $a_0, \dots, a_\ell$, where $\ell \le \Ham(S + \psi, S) + 2$.
		We then construct the strings $S_0, \dots, S_{\ell - 1}$, where $S_i$ is the restriction of $S$ to 
		$\set{u : u \in \d{S}, \h{u} \in (a_i, a_{i + 1}]}$.
	\end{proof}
\end{lemma}


\subsection{Partitioning of a periodic string}
The next theorem is used both to partition the pattern $P$ and the active text $\Ta$.

\tileDecomposition
\begin{proof}
We first construct in $\O(m^2)$ time a set $\Gamma \subseteq \Z^2$ such that $|\Gamma| = \O(\min\set{m, k})$ and $\forall u \in \Z^2 \; \exists \gamma \in \Gamma : u \equiv \gamma$ (\Cref{lattice_base}). Next, we use it to partition $R$ into a set of (truncated) subtile strings $\S$, such that $|\S| = \O(\min\set{m, k})$: Namely, for each $\gamma \in \Gamma$ (see \Cref{lattice_base}), we construct a (truncated) subtile string defined as the restriction of $R$ to $\d{R} \cap (\L + \gamma)$. For that, we consider each $u \in \d{R}$, construct its basis decomposition $u = s\cdot \phi + t\cdot \psi$, where $s,t\in \R$ by solving a system of linear equations in $\O(1)$ time, and add $u$ into the restriction corresponding to $\gamma = (s-\floor{s}) \cdot \phi + (t-\floor{t}) \cdot \psi \in \Gamma$. 

	We now apply \Cref{cut_partitioning} (\ref{partition_a}) to partition each $S \in \S$  into a set of (truncated) subtile strings $\S'$, such that $\S'$ partitions $R$ and $\Ham(S' + \phi, S') = 0$ for every $S' \in \S'$.
	Note that
	\[ |S'| = \sum_{S \in \S} \O(\Ham(S + \phi, S) + 1) = \O(\Ham(R + \phi, R) + |\S|) = \O(k),\]
	since $\phi$ is a $\O(k)$-period of $R$.
	Finally, we apply  \Cref{cut_partitioning} (\ref{partition_b}) to partition each $S' \in \S'$ into a set of (truncated) subtile strings $\S''$, such that $\S''$ partitions $R$ and $\Ham(S'' + \phi, S'') = 0$ and $\Ham(S'' + \psi, S'') = 0$ for every $S'' \in \S''$.
	Again we have
	\[ |S''| = \sum_{S' \in \S'} \O(\Ham(S' + \psi, S') + 1) = \O(\Ham(R + \psi, R) + |\S'|) = \O(k),\]
	since $R$ has an $\O(k)$-period $\psi$.
	since $\psi$ is a $\O(k)$-period of $R$.
	The red lines represent the partitioning done in the first phase, when constructing $\S'$, and blue in the second, when constructing $\S''$.
	By \Cref{monochromacy_condition}, the strings $S'' \in \S''$ are monochromatic.
	Apart from the $\O((\min\{k,m\})^2)$-time preprocessing of $\phi$ and $\psi$ during which the algorithm constructs $\Gamma$, the algorithm spends $\tO(|\d{R}| + k)$ time.
\end{proof}


\subsection{Text partitioning}
While \cref{tile_decomposition} is sufficient to partition $P$ (which has $\phi$ and $\psi$ as $\O(k)$-periods) into monochromatic truncated subtile strings, $T$ is not necessarily periodic, and we are obliged to extend \cref{tile_decomposition} further to obtain a partitioning for $T$. In this section, we show the following result:

\TextDecomposition

\subsection{Proof of \cref{text_decomposition}}
%We consider an empty set as a (triv) parallelogram.
%Also, we assume that a parallelogram contains the points laying on its border and its vertices.

\newcommand{\IQ}{\mathbb{R} \setminus \mathbb{Q}}
\begin{observation}\label{line_existence}
	For all $\ell \in \Z^+$ and $v \in \Z^2$ we can find a sequence of parallel lines $f_0, f_1, \dots, f_\ell$, where $f_i = \set{u : u \in \R^2, v \times u = c_i}$ for some $c_i \in \IQ$, such that
	\begin{itemize}
		\item $c_0 < v \times u < c_\ell$ for every $u \in [n]^2$, or namely, the set $[n]^2$ is between $f_0$ and $f_\ell$,
		\item $0 < c_{i + 1} - c_i = \O(n|v| / \ell)$ for every $i \in [\ell]$. (Geometrically, this property implies that the distance between every two consecutive lines is $\O(n / \ell)$.)
	\end{itemize}
\end{observation}

We use \Cref{line_existence} with $v = \phi$ to construct the lines $h_0, \dots, h_\ell$ and with $v = \psi$ to construct the lines $s_0, \dots, s_\ell$.
For every $i, j \in [\ell + 1]$ we construct a point $w_{i, j}$ as an intersection of $h_i$ and $s_j$ (since $\phi$ and $\psi$ are not colinear, $h_i$ and $s_j$ are not parallel).
For every $i, j \in [\ell]$ we construct a parallelogram $p_{i, j}$ with vertices $w_{i, j}, w_{i + 1, j}, w_{i + 1, j + 1}, w_{i, j + 1}$. 


Observe that every $u \in [n]^2$ is contained strictly in the interior of exactly one parallelogram $p_{i, j}$.

\begin{lemma}\label{monotonicity_lemma}
	For every $i \in [\ell - 1]$ and $j \in [\ell]$ we have
	\eq{
	\min \X(p_{i, j}) < \min \X(p_{i + 1, j}), \quad
	&\min \Y(p_{i, j}) \le \min \Y(p_{i + 1, j}), \\
	\max \X(p_{i, j}) < \max \X(p_{i + 1, j}), \quad
	&\max \Y(p_{i, j}) \le \max \Y(p_{i + 1, j})
	}
	and for every $i \in [\ell]$ and $j \in [\ell - 1]$ we have
	\eq{
		\min \X(p_{i, j}) \ge \min \X(p_{i, j + 1}), \quad
		&\min \Y(p_{i, j}) < \min \Y(p_{i, j + 1}), \\
		\max \X(p_{i, j}) \ge \max \X(p_{i, j + 1}), \quad
		&\max \Y(p_{i, j}) < \max \Y(p_{i, j + 1}).
	}
	\begin{proof}
		It follows from the fact that we selected $\phi \in [0, +\infty) \times (-\infty, 0)$ and $\psi \in (0, +\infty) \times [0, +\infty)$.
		For example, to prove the first inequality, we can consider a point $u \in p_{i + 1, j}$, such that $\x{u} = \min \X(p_{i + 1, j})$
		and then construct a point $v \in p_{i, j}$, such that $v = u - t\psi$ for some $t > 0$, and thus $\min \X(p_{i, j}) \le \x{v} < \x{u} = \min \X(p_{i + 1, j})$.
		The other inequalities can be proven analogously.
	\end{proof}
\end{lemma}

\begin{restatable*}{lemma}{DistanceBoundLemma}\label{distance_bound_lemma}
	For every $i, j \in [\ell]$ and every $u, v \in \X(p_{i, j}) \times \Y(p_{i, j})$, we have $|u - v| = \O(n / \ell)$.
\end{restatable*}
\begin{proof} See \Cref{distance_bound_lemma_proof}. \end{proof}

Consider the case when $\max \X(p_{i, j}) - \min \X(p_{i, j}) \ge m / 4$ for some $i, j \in [\ell]$.
By \Cref{distance_bound_lemma}, we would have $m / 4 \le \max \X(p_{i, j}) - \min \X(p_{i, j}) = \O(n / \ell)$, and thus $\ell = \O(1)$.
In that case we can return a trivial partitioning where $F = \Ta$ and the set of monochromatic strings is empty, since $\Ta$ is $\O(m)$-peripheral.
We can use the same argument if we have $\max \Y(p_{i, j}) - \min \Y(p_{i, j}) \ge m / 4$ for some $i, j \in [\ell]$.
Thus, from now on we will assume that $\max X_{i, j} - \min X_{i, j} < m / 4$ and $\max Y_{i, j} - \min Y_{i, j} < m / 4$ for every $i, j \in [\ell]$. 

Let $z = \frac{n - 1}{2}$. 
We split the plane with two lines $x = z$ and $y = z$ into four quarters:
\begin{enumerate}[1)]
	\item $K_1 = (z, +\infty) \times (z, +\infty)$,
	\item $K_2 = (-\infty, z) \times (z, +\infty)$,
	\item $K_3 = (-\infty, z) \times (-\infty, z)$,
	\item $K_4 = (z, +\infty) \times (-\infty, z)$.
\end{enumerate}
\newcommand{\I}{\mathcal{I}}
\newcommand{\G}{\mathcal{G}}
\newcommand{\C}{\mathcal{C}}
Let us denote by $\I$ the set of all parallelograms $p_{i, j}$, such that they intersect with the line $x = z$ or with the line $y = z$ (or both).
Observe that every parallelogram $p_{i, j} \not \in \I$ must be fully contained in one of the quarters, meaning $p_{i, j} \subseteq K_d$ for some $d \in \set{1, \dots, 4}$.

\begin{lemma}\label{I_size_bound}
	$|\I| = \O(\ell).$
	\begin{proof}
		Consider the line $x = z$, denoted $f$.
		It intersects with every line $h_0, \dots, h_\ell$ at most once (and does not overlap with any of them).
		Similarly, it intersects with every line $s_0, \dots, s_\ell$ at most once.
		Denote the set of such intersections as $U$.
		For every parallelogram $p \in \I$, there must exist $u \in U$, such that $u \in p$.
		For every $u \in U$, there are at most four parallelograms $p \in \I$, such that $u \in p$,
		thus the number of parallelograms intersecting with $f$ is at most $4|U| = \O(\ell)$.
		We can identically bound the number of parallelograms intersecting with the line $y = z$, and thus get $|\I| = \O(\ell)$.
	\end{proof}
\end{lemma}

Now consider any $j \in [\ell]$.
By \Cref{monotonicity_lemma}, we can find $s, t \in [\ell + 1]$, such that the sequence $p_{0, j}, \dots, p_{\ell - 1, j}$ is split into three groups:
\begin{enumerate}[a)]
	\item $p_{0, j}, \dots, p_{s - 1, j}$, which includes only parallelograms fully contained in $K_3$, \label{full in 3}
	\item $p_{s, j}, \dots, p_{t - 1, j}$, which does not include any parallelogram fully contained in $K_1$ or $K_3$,
	\item $p_{t, j}, \dots, p_{\ell - 1, j}$, which includes only parallelograms fully contained in $K_1$. \label{full in 1}
\end{enumerate}
We now ''merge together'' the parallelograms from group (\ref{full in 3}) and from group (\ref{full in 1}).
Specifically, we construct
\[
g^3_j = \bigcup_{i = 0}^{s - 1} p_{i, j}, \quad 
g^1_j = \bigcup_{i = t}^{\ell - 1} p_{i, j}.
\]
We do it for every $j \in [\ell]$.
Observe that the sets $g^1_0, \dots, g^1_{\ell - 1}$ are parallelograms (possibly empty) and that they cover the same area as all the fully contained in $K_1$ parallelograms $p_{i, j}$.
The same is true for $g^3_0, \dots, g^3_{\ell - 1}$ and the parallelograms in $K_3$. 

Now for every $i \in [\ell]$ we similarly find $s, t \in [\ell + 1]$, such that the sequence $p_{i, 0}, \dots, p_{i, \ell - 1}$ is split into three groups:
\begin{enumerate}[a)]
	\item $p_{i, 0}, \dots, p_{i, s - 1}$, which includes only parallelograms fully contained in $K_4$,
	\item $p_{i, s}, \dots, p_{i, t - 1}$, which does not include any parallelogram fully contained in $K_2$ or $K_4$,
	\item $p_{i, t}, \dots, p_{i, \ell - 1}$, which includes only parallelograms fully contained in $K_2$,
\end{enumerate}
and then construct
\[
g^4_i = \bigcup_{j = 0}^{s - 1} p_{i, j}, \quad 
g^2_i = \bigcup_{j = t}^{\ell - 1} p_{i, j}.
\]
We denote
\[
\G = \set{g^1_0, \dots, g^1_{\ell - 1}}  
\cup \set{g^2_0, \dots, g^2_{\ell - 1}}
\cup \set{g^3_0, \dots, g^3_{\ell - 1}}
\cup \set{g^4_0, \dots, g^4_{\ell - 1}}.
\]
Again observe that for every $u \in [n]^2$ there exists exactly one parallelogram $p \in \G \cup \I$, such that $u \in p$, and since the sides of $p$ do not contain integer points, $u$ lays strictly inside $p$.

\begin{figure}[!t]
	\begin{center}
		\includegraphics[width=0.8\textwidth]{drawings/text_decomposition}
	\end{center}
	\caption{The parallelograms from $\C$ (blue) and $\C'$ (green).}
	\label{figure:text_decomposition}
\end{figure}

\begin{definition}[Coverability]
	We say that a set $U \subseteq \Z^2$ is \textbf{coverable} if $U \subseteq \d{P + q}$ for some $q \in Q$.
\end{definition}

\begin{lemma}\label{I_division}
	For every $p \in \I$, the set $p \cap \Z^2$ is either coverable or $\O(n / \ell)$-peripheral.
	\begin{proof}
		Consider any $p \in \I$.
		Since the other cases are rotationally symmetric, assume that it intersects with some point $s \in \R^2$, such that $\x{s} = z$ and $\y{s} \ge z$.
		Let $v = (\floor{\max \X(p), \max \Y(p)})$.
		We have $\x{v} \ge \floor{z} = n / 2 - 1$ and $\y{v} \ge \floor{z} = n / 2 - 1$.
		If $v \not \in \d{\Ta}$, we can see that by \Cref{distance_bound_lemma},
		$|u - v| = \O(n / \ell)$ for every $u \in p \cap \Z^2$, thus $p$ is $\O(n / \ell)$-peripheral.
		If $v \in \d{\Ta}$, there exists $q \in Q$, such that $v \in [m]^2 + q$.
		Consider any $u \in p \cap \Z^2$.
		By the assumption that $\max \X(p) - \min \X(p) < m / 4$ and $\max \Y(p) - \min \Y(p) < m / 4$ we get
		\eq{
			\x{u} \ge n / 2 - m / 4 \ge n - m \ge \x{q},\\ 
			\y{u} \ge n / 2 - m / 4 \ge n - m \ge \y{q},
		}
		and since $\x{u} \le \x{v} \le \x{q} + m - 1$ and $\y{u} \le \y{v} \le \y{q} + m - 1$, we get $u \in [m]^2 + q$, thus $U \subseteq [m]^2 + q$.
	\end{proof}
\end{lemma}

\begin{lemma}\label{coverable is periodic}
	The restriction of $T$ to a coverable set has $\O(k)$-periods $\phi$ and $\psi$.
	\begin{proof}
		Let $R$ denote the restriction. For $q \in Q$, such that $\d{R} \subseteq \d{P + q}$ we have
		\eq{
			\Ham(R + \phi, R) \le \Ham(R + \phi, P + q + \phi) + \Ham(P + q + \phi, P + q) + \Ham(P + q, R) \le \\
			\le \Ham(T, P + q) + \Ham(P + \phi, P) + \Ham(P + q, T) = \O(k)
		}
		and identically $\Ham(R + \psi, R) = \O(k)$.
	\end{proof}
\end{lemma}

\begin{lemma}\label{parallelogram_split_lemma}
	For every $g \in \G$ we can construct two parallelograms $c$ and $b$, such that
	\begin{itemize}
		\item $c \cap \Z^2$ is coverable,
		\item $b \cap \Z^2$ is $\O(n / \ell)$-peripheral,
		\item $g \cap \Z^2$ is partitioned into $b \cap \Z^2$ and $c \cap \Z^2$.
	\end{itemize}
	\begin{proof} See the next section (\ref{parallelogram_split_lemma_proof}). \end{proof}
\end{lemma}

\newcommand{\B}{\mathcal{B}}

We split every non-empty parallelogram $g \in \G$ (by \Cref{parallelogram_split_lemma}) into parallelograms $c$ and $b$.
We construct the set $\C$ consisting of all the obtained parallelograms $c$ and a set $\B$ consisting of all the obtained parallelograms $b$.

We similarly divide the parallelograms in $\I$ (by \Cref{I_division}) and construct the sets $\C' = \set{p : p \in \I, p \cap \Z^2 \text{ is coverable}}$ and $\B' = \I \setminus \C'$.

Now construct $\U = \set{c \cap \Z^2 : c \in \C \cup \C'}$ and $V = \bigcup_{b \in \B \cup \B'} b \cap \d{\Ta}$.
Observe that all sets $U \in \U$ are coverable simple tiles, the set $V$ is $\O(n / \ell)$-peripheral, and $\d{\Ta}$ is partitioned into sets $\U \cup \set{V}$.

The decomposition is illustrated in \Cref{figure:text_decomposition}.
The points in the gray area are outside of the active text.
The parallelograms from $\C$ and $\C'$ are colored blue and green, respectively.
The points outside of them (in the white area) form the peripheral set $V$.

For each $U \in \U$ we construct the restriction of $T$ to $U$.
By \Cref{coverable is periodic}, it has $\O(k)$-periods $\phi$ and $\psi$, thus, by \Cref{tile_decomposition}, it can be partitioned into $\O(k)$ monochromatic simple subtile strings.
Since $|\C'| \le |\I| = \O(\ell)$ (by \Cref{I_size_bound}) and $|\C| \le |\G| = \O(\ell)$, we have $|\U| = |\C| + |\C'| = \O(\ell)$, thus the total number of constructed strings is $\O(\ell k)$.

Finally, we construct the restriction of $T$ to $V$, which is a $\O(n / \ell)$-peripheral string.


% PROOF OF SPLIT LEMMA
\subsubsection{Parallelogram splitting} \label{parallelogram_split_lemma_proof}

This section serves as the proof of \Cref{parallelogram_split_lemma}, introduced at the end of the previous section (\ref{text_decomposition_proof}).
Since for an empty parallelogram the proof is trivial, consider a non-empty set $g^1_j$ for some $j \in [\ell]$.
We will explore some properties of the part of the text contained in $K_1$ specifically, which then can be generalized to other quarters by symmetry.

\begin{lemma}\label{coverability_condition}
	Every set $U \subseteq K_1 \cap \Z^2$, such that $(\max \X(U), \max \Y(U)) \in \Ta$ is coverable.
	\begin{proof}
		Let $v = (\max \X(U), \max \Y(U))$.
		By assumption, there exists $q \in Q$, such that $v \in [m]^2 + q$.
		For every $u \in U$ we have
		$\x{q} \le n - m \le n / 2 \le \x{u} \le \x{v} < \x{q} + m$
		and $\y{q} \le n - m \le n / 2 \le \y{u} \le \y{v} < \y{q} + m$,
		thus $u \in [m]^2 + q$.
	\end{proof}
\end{lemma}

\begin{observation}\label{domination_lemma}
	By \Cref{coverability_condition}, there does not exist a pair of points $u \in \Z^2 \cap K_1 \setminus \d{\Ta}$ and $v \in \d{\Ta}$, such that $\x{u} \le \x{v}$ and $\y{u} \le \y{v}$. 
\end{observation}

Recall that there exists $t \in [\ell + 1]$, such that
$ g^1_j = \bigcup_{i = t}^{\ell - 1} p_{i, j}.$
We find 
\[ f = \min\set{i : i \in \set{t, \dots, \ell - 1}, (\floor{\max \X(p_{i, j})}, \floor{\max \Y(p_{i, j})}) \in \Z^2 \setminus \d{\Ta}}.\]
If the minimum does not exist, we consider $f = \ell$.
We then construct the parallelograms
\[c = \bigcup_{i = 0}^{f - 1} p_{i, j}, \quad b = \bigcup_{i = f}^{\ell - 1} p_{i, j}.\]

We now show that the set $c \cap \Z^2$ is coverable.
If $c \cap \Z^2$ is empty, then it is coverable, so let us assume it is not.
In that case $f > 0$.
It is clear that $c \cap \Z^2 \subseteq K_1$.
By \Cref{monotonicity_lemma}, we have 
\eq{
	\max \X(c) = \max \X(p_{f - 1, j}), \\
	\max \X(c) = \max \Y(p_{f - 1, j}),
}
and thus
\eq{
	\max \X(c \cap \Z^2) &\le \floor{\max \X(c)} = \floor{\max \X(p_{f - 1, j})}, \\
	\max \Y(c \cap \Z^2) &\le \floor{\max \Y(c)} = \floor{\max \Y(p_{f - 1, j})}.
}

We see that $(\max \X(c \cap \Z^2), \max \Y(c \cap \Z^2)) \in \d{\Ta}$, since it would otherwise contradict \Cref{domination_lemma},
considering that
$(\floor{\max \X(p_{f - 1, j})}, \floor{\max \Y(p_{f - 1, j})}) \in \d{\Ta}$.
We see that $c \cap \Z^2$ satisfies the conditions of \Cref{coverability_condition}, thus $c \cap \Z^2$ is coverable.

We now show that the set $b \cap \Z^2$ is $\O(n / \ell)$-peripheral.
If it is empty, then the proof is trivial, so let us assume it is not.
In that case $f < \ell$.
Denote $v = (\floor{\max \X(p_{f, j})}, \floor{\max \Y(p_{f, j})})$.
By definition, $v \in \Z^2 \setminus \d{\Ta}$.
Consider any point $u \in b \cap \Z^2$.
There exists exactly one $i \in \set{f, \dots, \ell - 1}$, such that $u$ lays strictly inside $p_{i, j}$.
Let $w = (\floor{\max \X(p_{i, j})}, \floor{\max \Y(p_{i, j})})$.
By \Cref{monotonicity_lemma}, we have $\x{w} \ge \x{v}$ and $\y{w} \ge \y{v}$, and by considering \Cref{domination_lemma},
we get $w \in \Z^2 \setminus \d{\Ta}$.
Finally, by \Cref{distance_bound_lemma}, we get $|u - w| = \O(n / \ell)$.

The constructions for $g^2_i$, $g^3_j$, $g^4_i$ are rotationally symmetric.


% PROOF OF DISTANCE BOUND
\subsubsection{Parallelogram span bounds} \label{distance_bound_lemma_proof}

In this section we will establish a distance bound for points laying inside or in the proximity of the constructed parallelograms.
Consider any fixed $p_{i, j}$ for some $i, j \in [\ell]$.
We first show some auxiliary (weaker) lemmas, which we then use to prove \Cref{distance_bound_lemma}.

\begin{lemma}\label{distance_bound_aux}
	For every $u, v \in p_{i, j}$, we have $|u - v| = \O(n / \ell)$.
	\begin{proof}
		Consider any $u, v \in p_{i, j}$ and denote $w = u - v$.
		By definition of $p_{i, j}$, we have
		\[ |\h{w}| = |\h{(u - v)}| = |\h{u} - \h{v}| = \O(n|\phi| / \ell) \]
		and similarly $|\s{w}| = \O(n|\psi| / \ell)$.
		Since $\phi$ and $\psi$ are not colinear, there exist $s, t \in \mathbb{R}$, such that $w = s\phi + t\psi$.
		Recall that by \Cref{get_periods} we have $|\phi \times \psi| \ge \frac{1}{2}|\phi||\psi|$ (since $|\sin \alpha| \ge 1/2$), thus
		\[ \frac{1}{2}|t||\phi||\psi| \le |t||\phi \times \psi| = |\phi \times (s\phi + t\psi)| = |\h{w}| = \O(n|\phi| / \ell), \]
		which gives us $|t\phi| = \O(n / \ell)$.
		We can similarly prove that $|s\psi| = \O(n / \ell)$ and finally
		\[ |w| = |s\phi + t\psi| \le |s\phi| + |t\psi| = \O(n / \ell). \]
	\end{proof}
\end{lemma}

\begin{lemma}\label{distance_bound_aux2}
	For every point $u \in \X(p_{i, j}) \times \Y(p_{i, j})$ there exists a point $v \in p_{i, j}$, such that $|u - v| = \O(n / \ell)$.
	\begin{proof}
		There exists a point $w \in p_{i, j}$, such that $\x{u} = \x{w}$, and a point $v \in p_{i, j}$, such that $\y{u} = \y{v}$.
		By \Cref{distance_bound_aux}, we have
		\[|u - v| = |\x{u} - \x{v}| = |\x{w} - \x{v}| \le |w - v| = \O(n / \ell).\] 
	\end{proof}
\end{lemma}

\DistanceBoundLemma
\begin{proof}
	Consider any $u, v \in p_{i, j}$.
	By \Cref{distance_bound_aux2}, there exist $u', v' \in p_{i, j}$ such that $|u - u'| = \O(n / \ell)$ and $|v - v'| = \O(n / \ell)$.
	By \Cref{distance_bound_aux}, we have
	\[ |u - v| \le |u - u'| + |u' - v'| + |v' - v| = \O(n / \ell). \]
\end{proof}


	
%%%%%%%%%%%%%%%%%%%%%%%%%%%%%%%%%%%%%%%%%%%%
%%%%%%%%%%%%%%%%%%%%%%%%%%%%%%%%%%%%%%%%%%%%
%%%%%%%%%%%%%%%%%%%%%%%%%%%%%%%%%%%%%%%%%%%%
\section{Algorithm}
\label{sec:algorithm}

\begin{figure}[!t]
	\begin{center}
		\includegraphics[width=0.8\textwidth]{drawings/quarter_split}
	\end{center}
	\caption{Construction of $S_1, \dots, S_4$. The red points represent $\d{S}$.}
	\label{figure:quarter_split}
\end{figure}
In this section we explore the properties of peripheral strings.
We consider any $d > 0$ and a non-empty $d$-peripheral string $S$, such that $\d{S} \subseteq \d{\Ta}$.
We define a partitioning of $S$ into strings $S_1, \dots, S_4$, by splitting it through the middle with a horizontal and vertical line. 
Specifically
\begin{itemize}
	\item $S_1$ is the restriction of $S$ to $\set{n / 2, \dots, n - 1} \times \set{n / 2, \dots, n - 1}$ (upper right quarter),
	\item $S_2$ is the restriction of $S$ to $\set{0, \dots, n / 2 - 1} \times \set{n / 2, \dots, n - 1}$ (upper left quarter),
	\item $S_3$ is the restriction of $S$ to $\set{0, \dots, n / 2 - 1} \times \set{0, \dots, n / 2 - 1}$ (lower left quarter),
	\item $S_4$ is the restriction of $S$ to $\set{n / 2, \dots, n - 1} \times \set{0, \dots, n / 2 - 1}$ (lower right quarter).
\end{itemize}
See \Cref{figure:quarter_split} for an illustration.
We will now demonstrate some characteristics of $S_1$, and by symmetry, generalize them to $S$.

\begin{lemma} \label{border_lemma}
	Assuming $d \le m/4$, there does not exist $u \in \d{S_1}$ and $v \in \d{\Ta}$ such that $\x{v} - \x{u} \ge d$ and $\y{v} - \y{u} \ge d$.
	\begin{proof}
		Assume the contrary.
		Since $u \in \d{S_1}$, the border distance of $u$ is at most $d$, so there exists $w \in \Z^2 \setminus \d{\Ta}$, such that
		$\x{u} - d \le \x{w} \le \x{u} + d$ and
		$\y{u} - d \le \y{w} \le \y{u} + d$.
		Since $v \in \d{T_a}$, there exists $q \in Q$ such that $v \in [m]^2 + q$.
		We have
		\[ \x{w} \ge \x{u} - d \ge n / 2 - m / 4 \ge n - m > \x{q} \]
		and
		\[ \x{w} \le \x{u} + d \le \x{v} \le \x{q} + m - 1. \]
		Similarly we can show that $\y{q} \le \y{w} \le \y{q} + m - 1$, and thus $w \in [m]^2 + q$.
		Since $[m]^2 + q \subseteq \d{\Ta}$ and $w \not \in \d{\Ta}$, we get a contradiction.
	\end{proof}
\end{lemma}

We now introduce two major theorems regarding peripheral strings, the first of which is proven in the next section (\ref{sigma_border_proof}):

\begin{restatable*}{theorem}{SigmaBorder}\label{sigma_border}
	We can calculate $\Ham(P + q, S)$ for every $q \in Q$ in total time $\tO(m^2 + md |\Sigma|)$, where $|\Sigma|$ is the number of different characters present in both $P$ and $S$.
\end{restatable*}

\DenseAlgo
\begin{proof}
	Recall the construction of the sets $\V_\sigma$ described in \Cref{periodicity_section}.
	We define $\sigma \in \Sigma$ to be a \textbf{frequent} character if $|\V_\sigma| \ge \sqrt{k}$ and if $|\V_\sigma| < \sqrt{k}$, we call it an \textbf{infrequent} character.
	We partition $S$ into two strings $F$ and $I$, based on character frequency,
	so that $F$ consists of only the frequent characters and $I$ consists of only the infrequent ones.
	For every $q \in Q$ we then have 
	\[\Ham(P + q, S) = \Ham(P + q, F) + \Ham(P + q, I).\]
	
	Observe that the number of different frequent characters is $\O(\sqrt{k})$, and thus, by \Cref{sigma_border}, we can calculate $\Ham(P + q, F)$ for every $q \in Q$ in total time $\tO(m^2 + mdk^{1/2})$, since $F$ is $d$-peripheral. 
	
	We partition $I$ into $|\d{I}|$ strings, one per every $u \in \d{I}$.
	Specifically, let $I_u$ be the restriction of $I$ to $\set{u}$ for every $u \in \d{I}$.
	We have $\Ham(P + q, I) = \sum_{u \in \d{I}} \Ham(P + q, I_u)$ for every $q \in Q$.
	By \Cref{subtile_definition}, $I_u$ are simple subtile strings, and thus, we can by \Cref{sparse_algo} calculate the results in $\tO(m^2 + \sum_{u \in \d{I}} |\V_{I(u)}|)$.
	Since $I(u)$ is an infrequent character for every $u \in \d{I}$, we have $|\V_{I(u)}| < k^{1/2}$ for every $u \in \d{I}$.
	By \Cref{area_bound} we have $|\d{I}| = \O(md)$, and thus the total complexity is $\tO(m^2 + mdk^{1/2})$.
\end{proof}

\subsubsection{Peripheral convolution} \label{sigma_border_proof}

This section serves as the proof of the theorem we just used to prove \Cref{dense_algo}:

\SigmaBorder

We base our approach on the simple method of calculating the Hamming distance by running an instance of FFT for each unique character.
We will again utilize partitioning to reduce the problem to some smaller ones and then solve them naively.
We will take advantage of the fact that the points close to the border can overlap only with a small subset of points from the pattern when considering the occurrences fully contained in the active text.

Recall that 
$\Ham(P + q, S) = \Ham(P + q, S_1) + \dots + \Ham(P + q, S_4)$.
We will only show how to calculate $\Ham(P + q, S_1)$ for every $q \in Q$, since the other cases are symmetric.
Consider a string $P_0$, defined as the restriction of $P$ to $[m - d]^2$ and a string $P_1$, defined as the restriction of $P$ to $\d{P} \setminus \d{P_0}$.
Since the strings $P_0$ and $P_1$ partition $P$, we have
\[ \Ham(P + q, S_1) = \Ham(P_0 + q, S_1) + \Ham(P_1 + q, S_1).\]


From now we will assume that $d \le m / 4$, since for $d > m / 4$ we can, by \Cref{general_fft}, calculate the results in time $\tO(m^2 + m^2|\Sigma|)$, which is sufficient.


\begin{lemma}\label{border_hamming_reduction}
	$\d{P_0 + q} \cap \d{S_1} = \emptyset$ for every $q \in Q$.
	\begin{proof}
		Let us assume the contrary.
		Select any $q \in Q$ such that $\d{P_0 + q} \cap \d{S_1}$ 
		contains some point $u$ and consider the point $v = (\x{u} + d, \y{u} + d)$.
		Since $u \in [m - d]^2 + q$, we have $v \in [m]^2 + q \subseteq \d{\Ta}$, thus the points $u \in \d{S_1}$ and $v \in \d{\Ta}$ contradict \Cref{border_lemma}.
	\end{proof}
\end{lemma}

\begin{observation}\label{border_hamming_split}
	$P_1$ can be partitioned into two strings $P_2$ and $P_3$ such that the width of $P_2$ and the height of $P_3$ are equal to $d$.
\end{observation}

By \Cref{border_hamming_reduction}, $\Ham(P_0 + q, S_1) = 0$ for every $q \in Q$
and by \Cref{border_hamming_split} we have 
\[\Ham(P + q, S_1) = \Ham(P_1 + q, S_1) = \Ham(P_2 + q, S_1) + \Ham(P_3 + q, S_1) \]
for some strings $P_2$ and $P_3$ partitioning $P_1$, such that the width of $P_2$ and the height of $P_3$ are equal to $d$.
We calculate $\Ham(P_2 + q, S_1)$ and $\Ham(P_3 + q, S_1)$ for every $Q$ independently and sum the results.
We only show how to calculate $\Ham(P_2 + q, S_1)$, since the other case is symmetric.

We will now partition $S_1$.
Consider a sequence of strings $U_0, \dots, U_{\lceil n / d \rceil - 1}$, where $U_i$ is the restriction of $S_1$ to $\set{id, \dots, id + d - 1} \times [n] \cap \d{S_1}$.
For the sake of formality (since the maximum/minimum of an empty set is undefined), let $V_0, \dots, V_{\ell - 1}$ consist of all non-empty strings $U_i$, given in the increasing order of $i$.
Observe that $V_0, \dots, V_{\ell - 1}$ partition $S_1$ and their width is not greater than $d$.

\begin{figure}[!t]
	\begin{center}
		\includegraphics[width=0.8\textwidth]{drawings/periphery_decomposition}
	\end{center}
	\caption{The decomposition of $S_1$.}
	\label{figure:periphery_decomposition}
\end{figure}

\begin{figure}[!t]
	\begin{center}
		\includegraphics[width=0.8\textwidth]{drawings/pattern_restriction}
	\end{center}
	\caption{Pattern partitioning.}
	\label{figure:pattern_restriction}
\end{figure}

For each $i \in [\ell]$ we find $h_i \in \Z^+$, which we define as the minimal number such that $(\x{v}, \y{u} + h_i) \not \in \d{\Ta}$ for every $u, v \in V_i$.
For better understanding, $h_i$ is an upper bound for the height of $V_i$.

The construction is illustrated in \Cref{figure:periphery_decomposition}.
The points in the gray area are outside of the active text.
The remaining ones are in the active text, where the red and green represent $\d{S_1}$, and the green belong to $\d{V_i}$ some fixed $i$.

\begin{lemma}\label{sum_of_h}
	The sum of all $h_i$ is $\O(m)$.
	\begin{proof}
		Since for $\ell < 2$ the proof is trivial, we assume $\ell \ge 2$.
		For every $i \in [\ell]$ (since $h_i$ is minimal) there exists a pair of points $u_i, v_i \in \d{V_i}$ such that $(\x{v_i}, \y{u_i} + h_i - 1) \in \d{\Ta}$.
		It can be shown that for all $i \ge 2$ we have
		\[ h_{i} \le \y{u_{i - 2}} - \y{u_{i}} + d,\]
		since if that was not the case for some $i$, then the points $u_{i - 2}$ and $(\x{v_{i}}, \y{u_{i}} + h_{i} - 1)$ would contradict \Cref{border_lemma}.
		We can conclude that
		\[
			\sum_{i = 0}^{\ell - 1} h_i
			\le h_0 + h_1 + \sum_{i = 2}^{\ell - 1} (\y{u_{i - 2}} - \y{u_i} + d)
			= h_0 + h_1 + \y{u_0} + \y{u_1} - \y{u_{\ell - 2}} - \y{u_{\ell - 1}} + (\ell - 2)d 
			= \O(m).
		\]
	\end{proof}
\end{lemma}

\begin{observation}\label{area_bound}
	By the above lemma $|\d{S_1}| = \O(md)$ and by extension $|\d{S}| = \O(md)$.
\end{observation}

For every $i \in [\ell]$ we construct the string $L_i$ as the restriction of $P_2$ to $[m] \times [m - h_i] \cap \d{P_2}$
and the string $H_i$ as the restriction of $P_2$ to $\d{P_2} \setminus \d{L_i}$.
The construction is illustrated in \Cref{figure:pattern_restriction}.
Since $L_i$ and $H_i$ partition $P_2$, we have
\[ \Ham(P_2 + q, S_1) = \sum_{i = 0}^{\ell - 1} \Ham(P_2 + q, V_i) = \sum_{i = 0}^{\ell - 1} \Ham(L_i + q, V_i) + \sum_{i = 0}^{\ell - 1} \Ham(H_i + q, V_i). \]

\begin{lemma}\label{pattern_height_reduction}
	$\d{L_i + q} \cap \d{V_i} = \emptyset$ for every $q \in Q$ and $i \in [\ell]$.
	\begin{proof}
		Let us assume the contrary.
		Select any $q \in Q$ and $i \in [\ell]$, such that $\d{L_i + q} \cap \d{V_i}$ 
		contains some point $u$ and consider the point $v = (\x{u}, \y{u} + h_i)$.
		Since $u \in [m] \times [m - h_i] + q$, we have $v \in [m]^2 + q \subseteq \d{\Ta}$, thus $v \in \d{\Ta}$, which contradicts the definition of $h_i$.
	\end{proof}
\end{lemma}

By \Cref{pattern_height_reduction}, for every $q \in Q$ we have $\sum_{i = 0}^{\ell - 1} \Ham(L_i + q, V_i) = 0$, thus our result is equal to $\sum_{i = 0}^{\ell - 1} \Ham(H_i + q, V_i)$.
We run the algorithm from \Cref{general_fft} for every pair of $H_i$ and $V_i$ and, since both $H_i$ and $V_i$ have widths not greater than $d$ and heights not greater than $h_i$, we obtain the total complexity of $\tO(\sum_{i = 0}^{\ell - 1} (|\Sigma| + 1)dh_i)$, 
which, by \Cref{sum_of_h}, is $\tO(m^2 + md|\Sigma|)$.




% SUBtile CONVOLUTION
\newcommand{\W}{\mathcal{W}}
\subsection{Subtile convolution} \label{subtile_convolution_proof}

Throughout this section we will denote $D = \set{u : u \in \L, \h{u} \ge 0, \s{u} \ge 0}$, where $\L$ is the set defined in \Cref{lattice_congruency}.
We start by introducing some auxiliary tools, which we later use in the proof of \Cref{sparse_algo}.

\begin{lemma}\label{primitive_conv}
	Given a set of subtiles $\V$ and a set of points $Q$, we can calculate
	\[ \sum_{V \in \V} |(D + q) \cap V| \]
	for every $q \in Q$ in total time $\tO(n^2 + |Q| + |\V|)$, assuming that every $V \in \V$ consists of vectors of length $\O(n)$.
	\begin{proof}
		For every $u \in \Z^2$ let us define $\score(u) = |\set{V : V \in \V, u \in V}|$. Observe that
		\[ \sum_{V \in \V} |(D + q) \cap V| = \sum_{u \in D + q} \score(u). \]
		We start by explicitly calculating the scores.
		We find the maximum length of a vector that some $V \in \V$ contains, which we denote $\ell$.
		We construct the set $U \subseteq \Z^2$ of all vectors of length at most $\ell$.
		By the assumption, we have $\ell = \O(n)$, and thus $|U| = \O(\ell ^ 2) = \O(n ^ 2)$.
		Since all the scores are zero for points outside of $U$, will only calculate them for $u \in U$.
		
		We find the set $\Gamma$ introduced in \Cref{lattice_base} and for every $\gamma \in \Gamma$ we construct $U_\gamma = U \cap (\L + \gamma)$.
		Consider any $u \in U_\gamma$ for some fixed $\gamma \in \Gamma$ and any $V \in \V$.
		Observe that if $V \not \equiv \gamma$, then $u \not \in V$ and thus $V$ does not contribute to $\score(u)$.
		If $V \equiv \gamma$, then we can find a tile $W$ such that $V = W \cap (\L + \gamma)$ and we have
		$u \in V \Leftrightarrow u \in W \cap (\L + \gamma) \Leftrightarrow u \in W$.
		Thus, if we denote $\W_\gamma$ as the set of tiles $W$ obtained for every $V \in \V$ such that $V \equiv \gamma$, then $\score(u)$ for $u \in U_\gamma$ is the number of tiles $W \in \W_\gamma$ such that $u \in W$.
		We calculate $\score(u)$ for every $u \in U_\gamma$ by sweeping $U_\gamma$ and $\W_\gamma$ in time $\tO(|U_\gamma| + |\W_\gamma|)$.
		We do it independently for every $\gamma \in \Gamma$, performing $\tO(|U| + |\V|) = \tO(n^2 + |\V|)$ operations in total.

		Now consider a query vector $q \in Q$.
		Let $\gamma \in \Gamma$ be such that $q \equiv \gamma$.
		We have already shown that the sum of scores for $u \in D + q$ is equal to the sum of scores for $u \in (D + q) \cap U$.
		Since $(D + q) \cap U = (D + q) \cap U_\gamma$, we see that the result is the sum of scores for such $u \in U_\gamma$, for which $\h{u} \ge \h{q}$ and $\s{u} \ge \s{q}$.
		If we denote $Q_\gamma = Q \cap (\L + \gamma)$, we see that we can calculate the results for all $q \in Q_\gamma$ by sweeping $Q_\gamma$ and $U_\gamma$ in time $\tO(|Q_\gamma| + |U_\gamma|)$.
		We do it independently for every $\gamma \in \Gamma$, performing $\tO(|Q| + |U|) = \tO(n^2 + |Q|)$ operations in total.
	\end{proof}
\end{lemma}

\begin{lemma}\label{primitive}
	For any simple subtile $U$ we can find $w_0, \dots, w_3 \in \Z^2$, such that
	\[ |U \cap X| = \sum_{j = 0}^3 (-1)^j |(D + w_j) \cap X|\]
	for every $X \subseteq \Z^2$.
	If $U$ consists of vectors of length $\O(n)$, then $w_0, \dots, w_3$ are of length $\O(n)$.
	\begin{proof}
		Let \eq{
			\phi_0 = \min\set{{\h{u} : u \in U}}, \quad 
			\phi_1 = \max\set{{\h{u} : u \in U}},\\
			\psi_0 = \min\set{{\s{u} : u \in U}}, \quad
			\psi_1 = \max\set{{\s{u} : u \in U}}.
		}
		Note that these values can be extracted from the signature.
		Since $U$ is a tile, there exist unique points $u_0, \dots, u_3 \in U$, such that
		\begin{itemize}
			\item $\h{u_0} = \phi_0$ and $\s{u_0} = \psi_0$,
			\item $\h{u_1} = \phi_1$ and $\s{u_1} = \psi_0$,
			\item $\h{u_2} = \phi_1$ and $\s{u_2} = \psi_1$,
			\item $\h{u_3} = \phi_0$ and $\s{u_3} = \psi_1$.
		\end{itemize}
		We construct \eq{
			w_0 = u_0, \quad 
			w_1 = u_1 + \psi, \quad
			w_2 = u_2 + \phi + \psi, \quad
			w_3 = u_3 + \phi.
		}
		It can be proven that the condition is satisfied.
	\end{proof}
\end{lemma}

\begin{theorem}\label{subtile_convolution}
	For a given list of signatures of simple subtiles $U_0, \dots, U_{\ell - 1}$, list of signatures of subtiles $V_0, \dots, V_{\ell - 1}$ and a set of vectors $Q$ we can calculate
	\[ \sum_{i = 0}^{\ell - 1} |(U_i + q) \cap V_i| \]
	for every $q \in Q$ in total time $\tO(m^2 + \ell + |Q|)$, assuming that the subtiles only contain vectors of length $\O(m)$.
	\begin{proof}
		We apply \Cref{primitive} to every $U_i$ and find $w_{i, 0}, \dots, w_{i, 3}$, so that we have
		\eq{
			\sum_{i = 0}^{\ell - 1}|(U_i + q) \cap V_i| 
			= \sum_{i = 0}^{\ell - 1}|U_i \cap (V_i - q)| 
			= \sum_{i = 0}^{\ell - 1} \sum_{j = 0}^3 (-1)^j |(D + w_{i, j}) \cap (V_i - q)| = \\
			= \sum_{j = 0}^3 (-1)^j \sum_{i = 0}^{\ell - 1} |(D + q) \cap (V_i - w_{i, j})|.
		}
		By \Cref{primitive_conv} we can independently calculate the values $\sum_{i = 0}^{\ell - 1} |(D + q) \cap (V_i - w_{i, j})|$ for every $j$ by running the algorithm for $\V_j = \set{V_i - w_{i, j} : i \in [\ell]}$ and $Q$.
	\end{proof}
\end{theorem}

\SparseAlgo
\begin{proof}
	Let $U = \bigcup_{S \in \S} \d{S}$. Observe that
	\[ \sum_{S \in \S} \Ham(P + q, S) = |\d{P + q} \cap U| - \sum_{S \in \S}\sum_{V \in \V_{\getchar{S}}} |\d{V + q} \cap \d{S}|.\]
	We can calculate $|\d{P + q} \cap U|$ for every $q \in Q$ with a single instance of FFT (see \Cref{sigman2d}) or by using prefix sums in time $\tO(m^2)$.
	To calculate the values
	\[ \sum_{S \in \S}\sum_{V \in \V_{\getchar{S}}} |\d{V + q} \cap \d{S}| \]
	we use the algorithm from \Cref{subtile_convolution} (where $\ell = \sum_{S \in \S} |\V_{\getchar{S}}|$).
\end{proof}















\bibliographystyle{plainurl}
\bibliography{references}

\end{document}

